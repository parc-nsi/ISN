
% Default to the notebook output style

    


% Inherit from the specified cell style.




    
\documentclass[11pt]{article}

    
    
    \usepackage[T1]{fontenc}
    % Nicer default font (+ math font) than Computer Modern for most use cases
    \usepackage{mathpazo}

    % Basic figure setup, for now with no caption control since it's done
    % automatically by Pandoc (which extracts ![](path) syntax from Markdown).
    \usepackage{graphicx}
    % We will generate all images so they have a width \maxwidth. This means
    % that they will get their normal width if they fit onto the page, but
    % are scaled down if they would overflow the margins.
    \makeatletter
    \def\maxwidth{\ifdim\Gin@nat@width>\linewidth\linewidth
    \else\Gin@nat@width\fi}
    \makeatother
    \let\Oldincludegraphics\includegraphics
    % Set max figure width to be 80% of text width, for now hardcoded.
    \renewcommand{\includegraphics}[1]{\Oldincludegraphics[width=.8\maxwidth]{#1}}
    % Ensure that by default, figures have no caption (until we provide a
    % proper Figure object with a Caption API and a way to capture that
    % in the conversion process - todo).
    \usepackage{caption}
    \DeclareCaptionLabelFormat{nolabel}{}
    \captionsetup{labelformat=nolabel}

    \usepackage{adjustbox} % Used to constrain images to a maximum size 
    \usepackage{xcolor} % Allow colors to be defined
    \usepackage{enumerate} % Needed for markdown enumerations to work
    \usepackage{geometry} % Used to adjust the document margins
    \usepackage{amsmath} % Equations
    \usepackage{amssymb} % Equations
    \usepackage{textcomp} % defines textquotesingle
    % Hack from http://tex.stackexchange.com/a/47451/13684:
    \AtBeginDocument{%
        \def\PYZsq{\textquotesingle}% Upright quotes in Pygmentized code
    }
    \usepackage{upquote} % Upright quotes for verbatim code
    \usepackage{eurosym} % defines \euro
    \usepackage[mathletters]{ucs} % Extended unicode (utf-8) support
    \usepackage[utf8x]{inputenc} % Allow utf-8 characters in the tex document
    \usepackage{fancyvrb} % verbatim replacement that allows latex
    \usepackage{grffile} % extends the file name processing of package graphics 
                         % to support a larger range 
    % The hyperref package gives us a pdf with properly built
    % internal navigation ('pdf bookmarks' for the table of contents,
    % internal cross-reference links, web links for URLs, etc.)
    \usepackage{hyperref}
    \usepackage{longtable} % longtable support required by pandoc >1.10
    \usepackage{booktabs}  % table support for pandoc > 1.12.2
    \usepackage[inline]{enumitem} % IRkernel/repr support (it uses the enumerate* environment)
    \usepackage[normalem]{ulem} % ulem is needed to support strikethroughs (\sout)
                                % normalem makes italics be italics, not underlines
    

    
    
    % Colors for the hyperref package
    \definecolor{urlcolor}{rgb}{0,.145,.698}
    \definecolor{linkcolor}{rgb}{.71,0.21,0.01}
    \definecolor{citecolor}{rgb}{.12,.54,.11}

    % ANSI colors
    \definecolor{ansi-black}{HTML}{3E424D}
    \definecolor{ansi-black-intense}{HTML}{282C36}
    \definecolor{ansi-red}{HTML}{E75C58}
    \definecolor{ansi-red-intense}{HTML}{B22B31}
    \definecolor{ansi-green}{HTML}{00A250}
    \definecolor{ansi-green-intense}{HTML}{007427}
    \definecolor{ansi-yellow}{HTML}{DDB62B}
    \definecolor{ansi-yellow-intense}{HTML}{B27D12}
    \definecolor{ansi-blue}{HTML}{208FFB}
    \definecolor{ansi-blue-intense}{HTML}{0065CA}
    \definecolor{ansi-magenta}{HTML}{D160C4}
    \definecolor{ansi-magenta-intense}{HTML}{A03196}
    \definecolor{ansi-cyan}{HTML}{60C6C8}
    \definecolor{ansi-cyan-intense}{HTML}{258F8F}
    \definecolor{ansi-white}{HTML}{C5C1B4}
    \definecolor{ansi-white-intense}{HTML}{A1A6B2}

    % commands and environments needed by pandoc snippets
    % extracted from the output of `pandoc -s`
    \providecommand{\tightlist}{%
      \setlength{\itemsep}{0pt}\setlength{\parskip}{0pt}}
    \DefineVerbatimEnvironment{Highlighting}{Verbatim}{commandchars=\\\{\}}
    % Add ',fontsize=\small' for more characters per line
    \newenvironment{Shaded}{}{}
    \newcommand{\KeywordTok}[1]{\textcolor[rgb]{0.00,0.44,0.13}{\textbf{{#1}}}}
    \newcommand{\DataTypeTok}[1]{\textcolor[rgb]{0.56,0.13,0.00}{{#1}}}
    \newcommand{\DecValTok}[1]{\textcolor[rgb]{0.25,0.63,0.44}{{#1}}}
    \newcommand{\BaseNTok}[1]{\textcolor[rgb]{0.25,0.63,0.44}{{#1}}}
    \newcommand{\FloatTok}[1]{\textcolor[rgb]{0.25,0.63,0.44}{{#1}}}
    \newcommand{\CharTok}[1]{\textcolor[rgb]{0.25,0.44,0.63}{{#1}}}
    \newcommand{\StringTok}[1]{\textcolor[rgb]{0.25,0.44,0.63}{{#1}}}
    \newcommand{\CommentTok}[1]{\textcolor[rgb]{0.38,0.63,0.69}{\textit{{#1}}}}
    \newcommand{\OtherTok}[1]{\textcolor[rgb]{0.00,0.44,0.13}{{#1}}}
    \newcommand{\AlertTok}[1]{\textcolor[rgb]{1.00,0.00,0.00}{\textbf{{#1}}}}
    \newcommand{\FunctionTok}[1]{\textcolor[rgb]{0.02,0.16,0.49}{{#1}}}
    \newcommand{\RegionMarkerTok}[1]{{#1}}
    \newcommand{\ErrorTok}[1]{\textcolor[rgb]{1.00,0.00,0.00}{\textbf{{#1}}}}
    \newcommand{\NormalTok}[1]{{#1}}
    
    % Additional commands for more recent versions of Pandoc
    \newcommand{\ConstantTok}[1]{\textcolor[rgb]{0.53,0.00,0.00}{{#1}}}
    \newcommand{\SpecialCharTok}[1]{\textcolor[rgb]{0.25,0.44,0.63}{{#1}}}
    \newcommand{\VerbatimStringTok}[1]{\textcolor[rgb]{0.25,0.44,0.63}{{#1}}}
    \newcommand{\SpecialStringTok}[1]{\textcolor[rgb]{0.73,0.40,0.53}{{#1}}}
    \newcommand{\ImportTok}[1]{{#1}}
    \newcommand{\DocumentationTok}[1]{\textcolor[rgb]{0.73,0.13,0.13}{\textit{{#1}}}}
    \newcommand{\AnnotationTok}[1]{\textcolor[rgb]{0.38,0.63,0.69}{\textbf{\textit{{#1}}}}}
    \newcommand{\CommentVarTok}[1]{\textcolor[rgb]{0.38,0.63,0.69}{\textbf{\textit{{#1}}}}}
    \newcommand{\VariableTok}[1]{\textcolor[rgb]{0.10,0.09,0.49}{{#1}}}
    \newcommand{\ControlFlowTok}[1]{\textcolor[rgb]{0.00,0.44,0.13}{\textbf{{#1}}}}
    \newcommand{\OperatorTok}[1]{\textcolor[rgb]{0.40,0.40,0.40}{{#1}}}
    \newcommand{\BuiltInTok}[1]{{#1}}
    \newcommand{\ExtensionTok}[1]{{#1}}
    \newcommand{\PreprocessorTok}[1]{\textcolor[rgb]{0.74,0.48,0.00}{{#1}}}
    \newcommand{\AttributeTok}[1]{\textcolor[rgb]{0.49,0.56,0.16}{{#1}}}
    \newcommand{\InformationTok}[1]{\textcolor[rgb]{0.38,0.63,0.69}{\textbf{\textit{{#1}}}}}
    \newcommand{\WarningTok}[1]{\textcolor[rgb]{0.38,0.63,0.69}{\textbf{\textit{{#1}}}}}
    
    
    % Define a nice break command that doesn't care if a line doesn't already
    % exist.
    \def\br{\hspace*{\fill} \\* }
    % Math Jax compatability definitions
    \def\gt{>}
    \def\lt{<}
    % Document parameters
    \title{Tris\_2018}
    
    
    

    % Pygments definitions
    
\makeatletter
\def\PY@reset{\let\PY@it=\relax \let\PY@bf=\relax%
    \let\PY@ul=\relax \let\PY@tc=\relax%
    \let\PY@bc=\relax \let\PY@ff=\relax}
\def\PY@tok#1{\csname PY@tok@#1\endcsname}
\def\PY@toks#1+{\ifx\relax#1\empty\else%
    \PY@tok{#1}\expandafter\PY@toks\fi}
\def\PY@do#1{\PY@bc{\PY@tc{\PY@ul{%
    \PY@it{\PY@bf{\PY@ff{#1}}}}}}}
\def\PY#1#2{\PY@reset\PY@toks#1+\relax+\PY@do{#2}}

\expandafter\def\csname PY@tok@sb\endcsname{\def\PY@tc##1{\textcolor[rgb]{0.73,0.13,0.13}{##1}}}
\expandafter\def\csname PY@tok@bp\endcsname{\def\PY@tc##1{\textcolor[rgb]{0.00,0.50,0.00}{##1}}}
\expandafter\def\csname PY@tok@vc\endcsname{\def\PY@tc##1{\textcolor[rgb]{0.10,0.09,0.49}{##1}}}
\expandafter\def\csname PY@tok@si\endcsname{\let\PY@bf=\textbf\def\PY@tc##1{\textcolor[rgb]{0.73,0.40,0.53}{##1}}}
\expandafter\def\csname PY@tok@cm\endcsname{\let\PY@it=\textit\def\PY@tc##1{\textcolor[rgb]{0.25,0.50,0.50}{##1}}}
\expandafter\def\csname PY@tok@s2\endcsname{\def\PY@tc##1{\textcolor[rgb]{0.73,0.13,0.13}{##1}}}
\expandafter\def\csname PY@tok@s1\endcsname{\def\PY@tc##1{\textcolor[rgb]{0.73,0.13,0.13}{##1}}}
\expandafter\def\csname PY@tok@s\endcsname{\def\PY@tc##1{\textcolor[rgb]{0.73,0.13,0.13}{##1}}}
\expandafter\def\csname PY@tok@ss\endcsname{\def\PY@tc##1{\textcolor[rgb]{0.10,0.09,0.49}{##1}}}
\expandafter\def\csname PY@tok@nb\endcsname{\def\PY@tc##1{\textcolor[rgb]{0.00,0.50,0.00}{##1}}}
\expandafter\def\csname PY@tok@fm\endcsname{\def\PY@tc##1{\textcolor[rgb]{0.00,0.00,1.00}{##1}}}
\expandafter\def\csname PY@tok@dl\endcsname{\def\PY@tc##1{\textcolor[rgb]{0.73,0.13,0.13}{##1}}}
\expandafter\def\csname PY@tok@nd\endcsname{\def\PY@tc##1{\textcolor[rgb]{0.67,0.13,1.00}{##1}}}
\expandafter\def\csname PY@tok@k\endcsname{\let\PY@bf=\textbf\def\PY@tc##1{\textcolor[rgb]{0.00,0.50,0.00}{##1}}}
\expandafter\def\csname PY@tok@mb\endcsname{\def\PY@tc##1{\textcolor[rgb]{0.40,0.40,0.40}{##1}}}
\expandafter\def\csname PY@tok@ch\endcsname{\let\PY@it=\textit\def\PY@tc##1{\textcolor[rgb]{0.25,0.50,0.50}{##1}}}
\expandafter\def\csname PY@tok@ne\endcsname{\let\PY@bf=\textbf\def\PY@tc##1{\textcolor[rgb]{0.82,0.25,0.23}{##1}}}
\expandafter\def\csname PY@tok@sa\endcsname{\def\PY@tc##1{\textcolor[rgb]{0.73,0.13,0.13}{##1}}}
\expandafter\def\csname PY@tok@gs\endcsname{\let\PY@bf=\textbf}
\expandafter\def\csname PY@tok@nt\endcsname{\let\PY@bf=\textbf\def\PY@tc##1{\textcolor[rgb]{0.00,0.50,0.00}{##1}}}
\expandafter\def\csname PY@tok@o\endcsname{\def\PY@tc##1{\textcolor[rgb]{0.40,0.40,0.40}{##1}}}
\expandafter\def\csname PY@tok@nc\endcsname{\let\PY@bf=\textbf\def\PY@tc##1{\textcolor[rgb]{0.00,0.00,1.00}{##1}}}
\expandafter\def\csname PY@tok@sh\endcsname{\def\PY@tc##1{\textcolor[rgb]{0.73,0.13,0.13}{##1}}}
\expandafter\def\csname PY@tok@ni\endcsname{\let\PY@bf=\textbf\def\PY@tc##1{\textcolor[rgb]{0.60,0.60,0.60}{##1}}}
\expandafter\def\csname PY@tok@mo\endcsname{\def\PY@tc##1{\textcolor[rgb]{0.40,0.40,0.40}{##1}}}
\expandafter\def\csname PY@tok@il\endcsname{\def\PY@tc##1{\textcolor[rgb]{0.40,0.40,0.40}{##1}}}
\expandafter\def\csname PY@tok@kc\endcsname{\let\PY@bf=\textbf\def\PY@tc##1{\textcolor[rgb]{0.00,0.50,0.00}{##1}}}
\expandafter\def\csname PY@tok@err\endcsname{\def\PY@bc##1{\setlength{\fboxsep}{0pt}\fcolorbox[rgb]{1.00,0.00,0.00}{1,1,1}{\strut ##1}}}
\expandafter\def\csname PY@tok@nv\endcsname{\def\PY@tc##1{\textcolor[rgb]{0.10,0.09,0.49}{##1}}}
\expandafter\def\csname PY@tok@vg\endcsname{\def\PY@tc##1{\textcolor[rgb]{0.10,0.09,0.49}{##1}}}
\expandafter\def\csname PY@tok@kd\endcsname{\let\PY@bf=\textbf\def\PY@tc##1{\textcolor[rgb]{0.00,0.50,0.00}{##1}}}
\expandafter\def\csname PY@tok@sd\endcsname{\let\PY@it=\textit\def\PY@tc##1{\textcolor[rgb]{0.73,0.13,0.13}{##1}}}
\expandafter\def\csname PY@tok@nl\endcsname{\def\PY@tc##1{\textcolor[rgb]{0.63,0.63,0.00}{##1}}}
\expandafter\def\csname PY@tok@c\endcsname{\let\PY@it=\textit\def\PY@tc##1{\textcolor[rgb]{0.25,0.50,0.50}{##1}}}
\expandafter\def\csname PY@tok@gu\endcsname{\let\PY@bf=\textbf\def\PY@tc##1{\textcolor[rgb]{0.50,0.00,0.50}{##1}}}
\expandafter\def\csname PY@tok@mf\endcsname{\def\PY@tc##1{\textcolor[rgb]{0.40,0.40,0.40}{##1}}}
\expandafter\def\csname PY@tok@mh\endcsname{\def\PY@tc##1{\textcolor[rgb]{0.40,0.40,0.40}{##1}}}
\expandafter\def\csname PY@tok@gp\endcsname{\let\PY@bf=\textbf\def\PY@tc##1{\textcolor[rgb]{0.00,0.00,0.50}{##1}}}
\expandafter\def\csname PY@tok@vi\endcsname{\def\PY@tc##1{\textcolor[rgb]{0.10,0.09,0.49}{##1}}}
\expandafter\def\csname PY@tok@gd\endcsname{\def\PY@tc##1{\textcolor[rgb]{0.63,0.00,0.00}{##1}}}
\expandafter\def\csname PY@tok@gi\endcsname{\def\PY@tc##1{\textcolor[rgb]{0.00,0.63,0.00}{##1}}}
\expandafter\def\csname PY@tok@c1\endcsname{\let\PY@it=\textit\def\PY@tc##1{\textcolor[rgb]{0.25,0.50,0.50}{##1}}}
\expandafter\def\csname PY@tok@kp\endcsname{\def\PY@tc##1{\textcolor[rgb]{0.00,0.50,0.00}{##1}}}
\expandafter\def\csname PY@tok@cs\endcsname{\let\PY@it=\textit\def\PY@tc##1{\textcolor[rgb]{0.25,0.50,0.50}{##1}}}
\expandafter\def\csname PY@tok@cp\endcsname{\def\PY@tc##1{\textcolor[rgb]{0.74,0.48,0.00}{##1}}}
\expandafter\def\csname PY@tok@se\endcsname{\let\PY@bf=\textbf\def\PY@tc##1{\textcolor[rgb]{0.73,0.40,0.13}{##1}}}
\expandafter\def\csname PY@tok@kr\endcsname{\let\PY@bf=\textbf\def\PY@tc##1{\textcolor[rgb]{0.00,0.50,0.00}{##1}}}
\expandafter\def\csname PY@tok@w\endcsname{\def\PY@tc##1{\textcolor[rgb]{0.73,0.73,0.73}{##1}}}
\expandafter\def\csname PY@tok@go\endcsname{\def\PY@tc##1{\textcolor[rgb]{0.53,0.53,0.53}{##1}}}
\expandafter\def\csname PY@tok@ow\endcsname{\let\PY@bf=\textbf\def\PY@tc##1{\textcolor[rgb]{0.67,0.13,1.00}{##1}}}
\expandafter\def\csname PY@tok@no\endcsname{\def\PY@tc##1{\textcolor[rgb]{0.53,0.00,0.00}{##1}}}
\expandafter\def\csname PY@tok@sc\endcsname{\def\PY@tc##1{\textcolor[rgb]{0.73,0.13,0.13}{##1}}}
\expandafter\def\csname PY@tok@ge\endcsname{\let\PY@it=\textit}
\expandafter\def\csname PY@tok@vm\endcsname{\def\PY@tc##1{\textcolor[rgb]{0.10,0.09,0.49}{##1}}}
\expandafter\def\csname PY@tok@gh\endcsname{\let\PY@bf=\textbf\def\PY@tc##1{\textcolor[rgb]{0.00,0.00,0.50}{##1}}}
\expandafter\def\csname PY@tok@gr\endcsname{\def\PY@tc##1{\textcolor[rgb]{1.00,0.00,0.00}{##1}}}
\expandafter\def\csname PY@tok@gt\endcsname{\def\PY@tc##1{\textcolor[rgb]{0.00,0.27,0.87}{##1}}}
\expandafter\def\csname PY@tok@cpf\endcsname{\let\PY@it=\textit\def\PY@tc##1{\textcolor[rgb]{0.25,0.50,0.50}{##1}}}
\expandafter\def\csname PY@tok@nn\endcsname{\let\PY@bf=\textbf\def\PY@tc##1{\textcolor[rgb]{0.00,0.00,1.00}{##1}}}
\expandafter\def\csname PY@tok@sr\endcsname{\def\PY@tc##1{\textcolor[rgb]{0.73,0.40,0.53}{##1}}}
\expandafter\def\csname PY@tok@sx\endcsname{\def\PY@tc##1{\textcolor[rgb]{0.00,0.50,0.00}{##1}}}
\expandafter\def\csname PY@tok@m\endcsname{\def\PY@tc##1{\textcolor[rgb]{0.40,0.40,0.40}{##1}}}
\expandafter\def\csname PY@tok@nf\endcsname{\def\PY@tc##1{\textcolor[rgb]{0.00,0.00,1.00}{##1}}}
\expandafter\def\csname PY@tok@mi\endcsname{\def\PY@tc##1{\textcolor[rgb]{0.40,0.40,0.40}{##1}}}
\expandafter\def\csname PY@tok@na\endcsname{\def\PY@tc##1{\textcolor[rgb]{0.49,0.56,0.16}{##1}}}
\expandafter\def\csname PY@tok@kn\endcsname{\let\PY@bf=\textbf\def\PY@tc##1{\textcolor[rgb]{0.00,0.50,0.00}{##1}}}
\expandafter\def\csname PY@tok@kt\endcsname{\def\PY@tc##1{\textcolor[rgb]{0.69,0.00,0.25}{##1}}}

\def\PYZbs{\char`\\}
\def\PYZus{\char`\_}
\def\PYZob{\char`\{}
\def\PYZcb{\char`\}}
\def\PYZca{\char`\^}
\def\PYZam{\char`\&}
\def\PYZlt{\char`\<}
\def\PYZgt{\char`\>}
\def\PYZsh{\char`\#}
\def\PYZpc{\char`\%}
\def\PYZdl{\char`\$}
\def\PYZhy{\char`\-}
\def\PYZsq{\char`\'}
\def\PYZdq{\char`\"}
\def\PYZti{\char`\~}
% for compatibility with earlier versions
\def\PYZat{@}
\def\PYZlb{[}
\def\PYZrb{]}
\makeatother


    % Exact colors from NB
    \definecolor{incolor}{rgb}{0.0, 0.0, 0.5}
    \definecolor{outcolor}{rgb}{0.545, 0.0, 0.0}



    
    % Prevent overflowing lines due to hard-to-break entities
    \sloppy 
    % Setup hyperref package
    \hypersetup{
      breaklinks=true,  % so long urls are correctly broken across lines
      colorlinks=true,
      urlcolor=urlcolor,
      linkcolor=linkcolor,
      citecolor=citecolor,
      }
    % Slightly bigger margins than the latex defaults
    
    \geometry{verbose,tmargin=1in,bmargin=1in,lmargin=1in,rmargin=1in}
    
    

    \begin{document}
    
    
    \maketitle
    
    

    
    \section{Algorithmes de tri}\label{algorithmes-de-tri}

    \subsection{Préambule}\label{pruxe9ambule}

    Quelques liens pour découvrir et comparer des algorithmes de tris :

\begin{itemize}
\tightlist
\item
  \href{fr.wikipedia.org/wiki/Algorithme_de_tri}{Article de Wikipedia}
\item
  \href{http://interstices.info/jcms/c_6973/les-algorithmes-de-tri}{Article
  du site Interstices}
\item
  \href{http://www.sorting-algorithms.com/}{Comparaison des algorithmes
  de tris}
\end{itemize}

Au fait ce texte saisi dans une cellule du
\href{http://ipython.org/}{notebook d'IPython} utilise le langage de
balises \href{http://daringfireball.net/projects/markdown/}{Markdown},
surcouche d'HTML.

    \subsection{Imports des modules}\label{imports-des-modules}

    \begin{Verbatim}[commandchars=\\\{\}]
{\color{incolor}In [{\color{incolor}11}]:} \PY{k+kn}{import} \PY{n+nn}{numpy} \PY{k}{as} \PY{n+nn}{np}
         \PY{k+kn}{import} \PY{n+nn}{matplotlib}\PY{n+nn}{.}\PY{n+nn}{pyplot} \PY{k}{as} \PY{n+nn}{plt}
         \PY{k+kn}{import} \PY{n+nn}{time}
         \PY{k+kn}{from} \PY{n+nn}{random} \PY{k}{import} \PY{n}{randint}
\end{Verbatim}


    \begin{Verbatim}[commandchars=\\\{\}]
{\color{incolor}In [{\color{incolor}12}]:} \PY{o}{\PYZpc{}}\PY{k}{matplotlib} inline
\end{Verbatim}


    \subsection{Tests de correction}\label{tests-de-correction}

    \begin{Verbatim}[commandchars=\\\{\}]
{\color{incolor}In [{\color{incolor}13}]:} \PY{c+c1}{\PYZsh{}Pour tester les fonctions de tri}
         \PY{c+c1}{\PYZsh{}un tableau contenant 3 tableaux d\PYZsq{}entiers aléatoires de tailles 10, 100, 1000}
         \PY{n}{BENCH2} \PY{o}{=} \PY{p}{[}\PY{p}{[}\PY{n}{randint}\PY{p}{(}\PY{l+m+mi}{1}\PY{p}{,} \PY{l+m+mi}{2}\PY{o}{*}\PY{l+m+mi}{10}\PY{o}{*}\PY{o}{*}\PY{n}{i}\PY{p}{)} \PY{k}{for} \PY{n}{\PYZus{}} \PY{o+ow}{in} \PY{n+nb}{range}\PY{p}{(}\PY{l+m+mi}{10}\PY{o}{*}\PY{o}{*}\PY{n}{i}\PY{p}{)}\PY{p}{]} \PY{k}{for} \PY{n}{i} \PY{o+ow}{in} \PY{n+nb}{range}\PY{p}{(}\PY{l+m+mi}{1}\PY{p}{,}\PY{l+m+mi}{4}\PY{p}{)}\PY{p}{]}
         
         \PY{c+c1}{\PYZsh{}idem mais avec 3 tableaux d\PYZsq{}entiers aléatoires de tailles impaires}
         \PY{n}{BENCH1} \PY{o}{=} \PY{p}{[}\PY{p}{[}\PY{n}{randint}\PY{p}{(}\PY{l+m+mi}{1}\PY{p}{,} \PY{l+m+mi}{2}\PY{o}{*}\PY{l+m+mi}{10}\PY{o}{*}\PY{o}{*}\PY{n}{i}\PY{p}{)} \PY{k}{for} \PY{n}{\PYZus{}} \PY{o+ow}{in} \PY{n+nb}{range}\PY{p}{(}\PY{l+m+mi}{10}\PY{o}{*}\PY{o}{*}\PY{n}{i} \PY{o}{+} \PY{l+m+mi}{1}\PY{p}{)}\PY{p}{]} \PY{k}{for} \PY{n}{i} \PY{o+ow}{in} \PY{n+nb}{range}\PY{p}{(}\PY{l+m+mi}{1}\PY{p}{,}\PY{l+m+mi}{4}\PY{p}{)}\PY{p}{]}
         
         \PY{k}{def} \PY{n+nf}{bontri}\PY{p}{(}\PY{n}{t}\PY{p}{)}\PY{p}{:}
             \PY{l+s+sd}{\PYZsq{}\PYZsq{}\PYZsq{}Détermine si un tableau est trié dans l\PYZsq{}ordre croissant\PYZsq{}\PYZsq{}\PYZsq{}}
             \PY{k}{for} \PY{n}{k} \PY{o+ow}{in} \PY{n+nb}{range}\PY{p}{(}\PY{n+nb}{len}\PY{p}{(}\PY{n}{t}\PY{p}{)}\PY{o}{\PYZhy{}}\PY{l+m+mi}{1}\PY{p}{)}\PY{p}{:}
                 \PY{k}{if} \PY{n}{t}\PY{p}{[}\PY{n}{k}\PY{p}{]} \PY{o}{\PYZgt{}} \PY{n}{t}\PY{p}{[}\PY{n}{k}\PY{o}{+}\PY{l+m+mi}{1}\PY{p}{]}\PY{p}{:}
                     \PY{k}{return} \PY{k+kc}{False}
             \PY{k}{return} \PY{k+kc}{True}
         
         \PY{k}{def} \PY{n+nf}{procedure\PYZus{}to\PYZus{}fonction}\PY{p}{(}\PY{n}{f}\PY{p}{)}\PY{p}{:}
             \PY{l+s+sd}{\PYZsq{}\PYZsq{}\PYZsq{}Remplace la fonction f qui est une procedure ne retournant rien par une }
         \PY{l+s+sd}{    fonction fbis qui exécute f sur ses arguments puis retourne ses arguments.}
         \PY{l+s+sd}{    Nécessaire pour composer une fonction de tri sur place avec bontri\PYZsq{}\PYZsq{}\PYZsq{}}
             
             \PY{k}{def} \PY{n+nf}{fbis}\PY{p}{(}\PY{o}{*}\PY{n}{args}\PY{p}{)}\PY{p}{:}
                 \PY{n}{f}\PY{p}{(}\PY{o}{*}\PY{n}{args}\PY{p}{)}
                 \PY{k}{return} \PY{n}{args}
                 
             \PY{k}{return} \PY{n}{fbis}
         
         \PY{k}{def} \PY{n+nf}{test\PYZus{}tri}\PY{p}{(}\PY{n}{tri}\PY{p}{,} \PY{n}{BENCH}\PY{p}{)}\PY{p}{:}
             \PY{c+c1}{\PYZsh{}copie profonde de BENCH qui est un tableau de tableaux}
             \PY{n}{COPIE} \PY{o}{=} \PY{p}{[}\PY{n}{t}\PY{p}{[}\PY{p}{:}\PY{p}{]} \PY{k}{for} \PY{n}{t} \PY{o+ow}{in} \PY{n}{BENCH}\PY{p}{]} 
             \PY{k}{return} \PY{p}{[}\PY{n}{bontri}\PY{p}{(}\PY{n}{procedure\PYZus{}to\PYZus{}fonction}\PY{p}{(}\PY{n}{tri}\PY{p}{)}\PY{p}{(}\PY{n}{t}\PY{p}{)}\PY{p}{)} \PY{k}{for} \PY{n}{t} \PY{o+ow}{in} \PY{n}{COPIE}\PY{p}{]}
\end{Verbatim}


    \subsubsection{Chronomètre}\label{chronomuxe8tre}

    \begin{Verbatim}[commandchars=\\\{\}]
{\color{incolor}In [{\color{incolor}14}]:} \PY{k}{def} \PY{n+nf}{timetest}\PY{p}{(}\PY{n}{fonction}\PY{p}{)}\PY{p}{:}
             \PY{l+s+sd}{\PYZdq{}\PYZdq{}\PYZdq{}exécute la fonction et affiche son temps d\PYZsq{}exécution \PYZdq{}\PYZdq{}\PYZdq{}}
             
             \PY{k}{def} \PY{n+nf}{fonction\PYZus{}modifiee}\PY{p}{(}\PY{o}{*}\PY{n}{args}\PY{p}{,}\PY{o}{*}\PY{o}{*}\PY{n}{kargs}\PY{p}{)}\PY{p}{:}
                 \PY{n}{debut} \PY{o}{=} \PY{n}{time}\PY{o}{.}\PY{n}{perf\PYZus{}counter}\PY{p}{(}\PY{p}{)}
                 \PY{n}{fonction}\PY{p}{(}\PY{o}{*}\PY{n}{args}\PY{p}{,}\PY{o}{*}\PY{o}{*}\PY{n}{kargs}\PY{p}{)}
                 \PY{k}{return} \PY{n}{time}\PY{o}{.}\PY{n}{perf\PYZus{}counter}\PY{p}{(}\PY{p}{)} \PY{o}{\PYZhy{}} \PY{n}{debut}
             
             \PY{k}{return} \PY{n}{fonction\PYZus{}modifiee}
\end{Verbatim}


    \subsection{Tri par sélection}\label{tri-par-suxe9lection}

    \subsubsection{Implémentations du tri par
sélection}\label{impluxe9mentations-du-tri-par-suxe9lection}

    \begin{Verbatim}[commandchars=\\\{\}]
{\color{incolor}In [{\color{incolor}15}]:} \PY{k}{def} \PY{n+nf}{index\PYZus{}mini}\PY{p}{(}\PY{n}{t}\PY{p}{,} \PY{n}{debut}\PY{p}{)}\PY{p}{:}   
             \PY{k}{if} \PY{n}{t} \PY{o}{==} \PY{p}{[}\PY{p}{]}\PY{p}{:}
                 \PY{k}{return} \PY{k+kc}{None}
             \PY{n}{imini} \PY{o}{=} \PY{n}{debut}
             \PY{k}{for} \PY{n}{j} \PY{o+ow}{in} \PY{n+nb}{range}\PY{p}{(}\PY{n}{debut} \PY{o}{+} \PY{l+m+mi}{1}\PY{p}{,} \PY{n+nb}{len}\PY{p}{(}\PY{n}{t}\PY{p}{)}\PY{p}{)}\PY{p}{:}
                 \PY{k}{if} \PY{n}{t}\PY{p}{[}\PY{n}{j}\PY{p}{]} \PY{o}{\PYZlt{}} \PY{n}{t}\PY{p}{[}\PY{n}{imini}\PY{p}{]}\PY{p}{:}
                     \PY{n}{imini} \PY{o}{=} \PY{n}{j}
             \PY{k}{return} \PY{n}{imini} 
         
         \PY{k}{def} \PY{n+nf}{tri\PYZus{}selection}\PY{p}{(}\PY{n}{t}\PY{p}{)}\PY{p}{:}
             \PY{l+s+sd}{\PYZsq{}\PYZsq{}\PYZsq{}Tri par sélection avec sélection du minimum\PYZsq{}\PYZsq{}\PYZsq{}}
             \PY{n}{n} \PY{o}{=} \PY{n+nb}{len}\PY{p}{(}\PY{n}{t}\PY{p}{)} 
             \PY{k}{for} \PY{n}{debut} \PY{o+ow}{in} \PY{n+nb}{range}\PY{p}{(}\PY{n}{n} \PY{o}{\PYZhy{}} \PY{l+m+mi}{1}\PY{p}{)}\PY{p}{:}
                 \PY{n}{imini} \PY{o}{=} \PY{n}{index\PYZus{}mini}\PY{p}{(}\PY{n}{t}\PY{p}{,} \PY{n}{debut}\PY{p}{)}
                 \PY{n}{t}\PY{p}{[}\PY{n}{debut}\PY{p}{]}\PY{p}{,} \PY{n}{t}\PY{p}{[}\PY{n}{imini}\PY{p}{]} \PY{o}{=} \PY{n}{t}\PY{p}{[}\PY{n}{imini}\PY{p}{]}\PY{p}{,} \PY{n}{t}\PY{p}{[}\PY{n}{debut}\PY{p}{]}
                 
         \PY{k}{def} \PY{n+nf}{index\PYZus{}maxi}\PY{p}{(}\PY{n}{t}\PY{p}{,} \PY{n}{fin}\PY{p}{)}\PY{p}{:}   
             \PY{k}{if} \PY{n}{t} \PY{o}{==} \PY{p}{[}\PY{p}{]}\PY{p}{:}
                 \PY{k}{return} \PY{k+kc}{None}
             \PY{n}{imaxi} \PY{o}{=} \PY{n}{fin}
             \PY{k}{for} \PY{n}{j} \PY{o+ow}{in} \PY{n+nb}{range}\PY{p}{(}\PY{l+m+mi}{0}\PY{p}{,} \PY{n}{fin}\PY{p}{)}\PY{p}{:}
                 \PY{k}{if} \PY{n}{t}\PY{p}{[}\PY{n}{j}\PY{p}{]} \PY{o}{\PYZgt{}} \PY{n}{t}\PY{p}{[}\PY{n}{imaxi}\PY{p}{]}\PY{p}{:}
                     \PY{n}{imaxi} \PY{o}{=} \PY{n}{j}
             \PY{k}{return} \PY{n}{imaxi}
         
         \PY{k}{def} \PY{n+nf}{tri\PYZus{}selection2}\PY{p}{(}\PY{n}{t}\PY{p}{)}\PY{p}{:}
             \PY{l+s+sd}{\PYZsq{}\PYZsq{}\PYZsq{}Tri par sélection avec sélection du maximum\PYZsq{}\PYZsq{}\PYZsq{}}
             \PY{n}{n} \PY{o}{=} \PY{n+nb}{len}\PY{p}{(}\PY{n}{t}\PY{p}{)} 
             \PY{k}{for} \PY{n}{fin} \PY{o+ow}{in} \PY{n+nb}{range}\PY{p}{(}\PY{n}{n} \PY{o}{\PYZhy{}} \PY{l+m+mi}{1}\PY{p}{,} \PY{l+m+mi}{0}\PY{p}{,} \PY{o}{\PYZhy{}}\PY{l+m+mi}{1}\PY{p}{)}\PY{p}{:}
                 \PY{n}{imaxi} \PY{o}{=} \PY{n}{index\PYZus{}maxi}\PY{p}{(}\PY{n}{t}\PY{p}{,} \PY{n}{fin}\PY{p}{)}
                 \PY{n}{t}\PY{p}{[}\PY{n}{fin}\PY{p}{]}\PY{p}{,} \PY{n}{t}\PY{p}{[}\PY{n}{imaxi}\PY{p}{]} \PY{o}{=} \PY{n}{t}\PY{p}{[}\PY{n}{imaxi}\PY{p}{]}\PY{p}{,} \PY{n}{t}\PY{p}{[}\PY{n}{fin}\PY{p}{]}
\end{Verbatim}


    \subsubsection{Test de correction du tri par
sélection}\label{test-de-correction-du-tri-par-suxe9lection}

    \begin{Verbatim}[commandchars=\\\{\}]
{\color{incolor}In [{\color{incolor}16}]:} \PY{k}{for} \PY{n}{tri} \PY{o+ow}{in} \PY{p}{[}\PY{n}{tri\PYZus{}selection}\PY{p}{,} \PY{n}{tri\PYZus{}selection2}\PY{p}{]}\PY{p}{:}
             \PY{n+nb}{print}\PY{p}{(}\PY{n}{test\PYZus{}tri}\PY{p}{(}\PY{n}{tri}\PY{p}{,} \PY{n}{BENCH1}\PY{p}{)}\PY{p}{)}
             \PY{n+nb}{print}\PY{p}{(}\PY{n}{test\PYZus{}tri}\PY{p}{(}\PY{n}{tri}\PY{p}{,} \PY{n}{BENCH2}\PY{p}{)}\PY{p}{)}
\end{Verbatim}


    \begin{Verbatim}[commandchars=\\\{\}]
[True, True, True]
[True, True, True]
[True, True, True]
[True, True, True]

    \end{Verbatim}

    \subsubsection{Complexité du tri par
sélection}\label{complexituxe9-du-tri-par-suxe9lection}

    Complexité du tri par sélection pour une liste de \(n\) entiers : chaque
boucle interne de recherche du maximum dans la partie de \(n-i\) valeurs
non encore triées, effectue \(n-i\) comparaisons ; donc l'algorithme
nécessite \((n-1)+(n-2)+\cdots+1=\frac{n(n-1)}{2}\) comparaisons. Il
s'agit donc d'une \textbf{complexité quadratique} que la liste initiale
soit déjà triée ou non (comme on peut le constater sur l'exemple
ci-dessus).

\begin{itemize}
\tightlist
\item
  \href{http://fr.wikipedia.org/wiki/Tri_par_sélection}{Article de
  Wikipedia}
\item
  \href{http://www.sorting-algorithms.com/selection-sort}{Le site
  sorting algorithms}
\end{itemize}

    \begin{Verbatim}[commandchars=\\\{\}]
{\color{incolor}In [{\color{incolor}46}]:} \PY{n}{liste\PYZus{}tris} \PY{o}{=} \PY{p}{[}\PY{n}{tri\PYZus{}selection}\PY{p}{,} \PY{n}{tri\PYZus{}selection2}\PY{p}{]}
         \PY{n}{liste\PYZus{}taille} \PY{o}{=} \PY{n+nb}{list}\PY{p}{(}\PY{n+nb}{range}\PY{p}{(}\PY{l+m+mi}{10}\PY{p}{,} \PY{l+m+mi}{1001}\PY{p}{,} \PY{l+m+mi}{10}\PY{p}{)}\PY{p}{)}
         \PY{n}{liste\PYZus{}rapport\PYZus{}carre} \PY{o}{=} \PY{n}{np}\PY{o}{.}\PY{n}{array}\PY{p}{(}\PY{p}{[}\PY{p}{[}\PY{l+m+mi}{0} \PY{k}{for} \PY{n}{\PYZus{}} \PY{o+ow}{in} \PY{n+nb}{range}\PY{p}{(}\PY{n+nb}{len}\PY{p}{(}\PY{n}{liste\PYZus{}taille}\PY{p}{)}\PY{p}{)}\PY{p}{]} \PY{k}{for} \PY{n}{\PYZus{}} \PY{o+ow}{in} \PY{n+nb}{range}\PY{p}{(}\PY{n+nb}{len}\PY{p}{(}\PY{n}{liste\PYZus{}tris}\PY{p}{)}\PY{p}{)}\PY{p}{]}\PY{p}{,} \PY{n}{dtype}\PY{o}{=}\PY{l+s+s1}{\PYZsq{}}\PY{l+s+s1}{float}\PY{l+s+s1}{\PYZsq{}}\PY{p}{)}
         \PY{k}{for} \PY{n}{i}\PY{p}{,} \PY{n}{taille} \PY{o+ow}{in} \PY{n+nb}{enumerate}\PY{p}{(}\PY{n}{liste\PYZus{}taille}\PY{p}{)}\PY{p}{:}
             \PY{n}{liste} \PY{o}{=} \PY{p}{[}\PY{n}{randint}\PY{p}{(}\PY{l+m+mi}{0}\PY{p}{,} \PY{n}{taille}\PY{p}{)} \PY{k}{for} \PY{n}{\PYZus{}} \PY{o+ow}{in} \PY{n+nb}{range}\PY{p}{(}\PY{n}{taille}\PY{p}{)}\PY{p}{]}
             \PY{k}{for} \PY{n}{j}\PY{p}{,} \PY{n}{tri} \PY{o+ow}{in} \PY{n+nb}{enumerate}\PY{p}{(}\PY{n}{liste\PYZus{}tris}\PY{p}{)}\PY{p}{:}             
                 \PY{n}{liste\PYZus{}rapport\PYZus{}carre}\PY{p}{[}\PY{n}{j}\PY{p}{]}\PY{p}{[}\PY{n}{i}\PY{p}{]} \PY{o}{=} \PY{n}{timetest}\PY{p}{(}\PY{n}{tri}\PY{p}{)}\PY{p}{(}\PY{n}{liste}\PY{p}{[}\PY{p}{:}\PY{p}{]}\PY{p}{)} \PY{o}{/} \PY{p}{(}\PY{n}{liste\PYZus{}taille}\PY{p}{[}\PY{n}{i}\PY{p}{]}\PY{p}{)} \PY{o}{*}\PY{o}{*} \PY{l+m+mi}{2}        
         \PY{k}{for} \PY{n}{k}\PY{p}{,} \PY{n}{rapport\PYZus{}carre} \PY{o+ow}{in} \PY{n+nb}{enumerate}\PY{p}{(}\PY{n}{liste\PYZus{}rapport\PYZus{}carre}\PY{p}{)}\PY{p}{:}
             \PY{n}{plt}\PY{o}{.}\PY{n}{plot}\PY{p}{(}\PY{n}{liste\PYZus{}taille}\PY{p}{,} \PY{n}{rapport\PYZus{}carre} \PY{o}{/} \PY{n}{rapport\PYZus{}carre}\PY{o}{.}\PY{n}{mean}\PY{p}{(}\PY{p}{)}\PY{p}{,} \PY{n}{label}\PY{o}{=}\PY{n}{liste\PYZus{}tris}\PY{p}{[}\PY{n}{k}\PY{p}{]}\PY{o}{.}\PY{n+nv+vm}{\PYZus{}\PYZus{}name\PYZus{}\PYZus{}}\PY{p}{,} \PY{n}{marker}\PY{o}{=}\PY{l+s+s1}{\PYZsq{}}\PY{l+s+s1}{o}\PY{l+s+s1}{\PYZsq{}}\PY{p}{)}
         \PY{n}{plt}\PY{o}{.}\PY{n}{title}\PY{p}{(}\PY{l+s+sa}{r}\PY{l+s+s2}{\PYZdq{}}\PY{l+s+s2}{Temps / taille ** 2 normalisé par la moyenne}\PY{l+s+s2}{\PYZdq{}}\PY{p}{)}
         \PY{n}{plt}\PY{o}{.}\PY{n}{legend}\PY{p}{(}\PY{p}{)}
         \PY{n}{plt}\PY{o}{.}\PY{n}{savefig}\PY{p}{(}\PY{l+s+s1}{\PYZsq{}}\PY{l+s+s1}{complexite\PYZhy{}quadratique\PYZhy{}tri\PYZus{}selection.png}\PY{l+s+s1}{\PYZsq{}}\PY{p}{)}
\end{Verbatim}


    \begin{center}
    \adjustimage{max size={0.9\linewidth}{0.9\paperheight}}{output_17_0.png}
    \end{center}
    { \hspace*{\fill} \\}
    
    \subsection{Tri par bulles}\label{tri-par-bulles}

    L'algorithme du tri par bulles consiste à trier un tableau en ne
s'autorisant qu'à échanger deux éléments consécutifs de ce tableau. On
peut démontrer que l'algorithme suivant trie n'importe quel tableau :

\begin{itemize}
\tightlist
\item
  chercher deux éléments consécutifs rangés dans le désordre ;
\item
  si deux tels éléments existent, les échanger et recommencer ;
\item
  sinon arrêter.
\end{itemize}

Descriptions du tri par bulles :

\begin{itemize}
\tightlist
\item
  \href{http://fr.wikipedia.org/wiki/Tri_\%C3\%A0_bulles}{Article de
  Wikipedia}
\item
  \href{http://www.sorting-algorithms.com/bubble-sort}{Le site sorting
  algorithms}
\end{itemize}

    \subsubsection{Implémentations du tri par
bulles}\label{impluxe9mentations-du-tri-par-bulles}

    \begin{Verbatim}[commandchars=\\\{\}]
{\color{incolor}In [{\color{incolor}17}]:} \PY{k}{def} \PY{n+nf}{tri\PYZus{}bulles}\PY{p}{(}\PY{n}{t}\PY{p}{)}\PY{p}{:}
             \PY{n}{n} \PY{o}{=} \PY{n+nb}{len}\PY{p}{(}\PY{n}{t}\PY{p}{)}
             \PY{k}{for} \PY{n}{bulle} \PY{o+ow}{in} \PY{n+nb}{range}\PY{p}{(}\PY{n}{n} \PY{o}{\PYZhy{}} \PY{l+m+mi}{1}\PY{p}{)}\PY{p}{:}
                 \PY{k}{for} \PY{n}{k} \PY{o+ow}{in} \PY{n+nb}{range}\PY{p}{(}\PY{l+m+mi}{0}\PY{p}{,} \PY{n}{n} \PY{o}{\PYZhy{}} \PY{l+m+mi}{1} \PY{o}{\PYZhy{}} \PY{n}{bulle}\PY{p}{)}\PY{p}{:}
                     \PY{k}{if} \PY{n}{t}\PY{p}{[}\PY{n}{k} \PY{o}{+} \PY{l+m+mi}{1}\PY{p}{]} \PY{o}{\PYZlt{}} \PY{n}{t}\PY{p}{[}\PY{n}{k}\PY{p}{]}\PY{p}{:}
                         \PY{n}{t}\PY{p}{[}\PY{n}{k} \PY{o}{+} \PY{l+m+mi}{1}\PY{p}{]}\PY{p}{,} \PY{n}{t}\PY{p}{[}\PY{n}{k}\PY{p}{]} \PY{o}{=} \PY{n}{t}\PY{p}{[}\PY{n}{k}\PY{p}{]}\PY{p}{,} \PY{n}{t}\PY{p}{[}\PY{n}{k} \PY{o}{+} \PY{l+m+mi}{1}\PY{p}{]}
                         
                         
         \PY{k}{def} \PY{n+nf}{tri\PYZus{}bulles2}\PY{p}{(}\PY{n}{t}\PY{p}{)}\PY{p}{:}
             \PY{n}{n} \PY{o}{=} \PY{n+nb}{len}\PY{p}{(}\PY{n}{t}\PY{p}{)}
             \PY{n}{continuer} \PY{o}{=} \PY{k+kc}{True}
             \PY{n}{bulle} \PY{o}{=} \PY{l+m+mi}{0}
             \PY{k}{while} \PY{n}{continuer}\PY{p}{:}
                 \PY{n}{continuer} \PY{o}{=} \PY{k+kc}{False}
                 \PY{k}{for} \PY{n}{k} \PY{o+ow}{in} \PY{n+nb}{range}\PY{p}{(}\PY{l+m+mi}{0}\PY{p}{,} \PY{n}{n} \PY{o}{\PYZhy{}} \PY{l+m+mi}{1} \PY{o}{\PYZhy{}} \PY{n}{bulle}\PY{p}{)}\PY{p}{:}
                     \PY{k}{if} \PY{n}{t}\PY{p}{[}\PY{n}{k} \PY{o}{+} \PY{l+m+mi}{1}\PY{p}{]} \PY{o}{\PYZlt{}} \PY{n}{t}\PY{p}{[}\PY{n}{k}\PY{p}{]}\PY{p}{:}
                         \PY{n}{t}\PY{p}{[}\PY{n}{k} \PY{o}{+} \PY{l+m+mi}{1}\PY{p}{]}\PY{p}{,} \PY{n}{t}\PY{p}{[}\PY{n}{k}\PY{p}{]} \PY{o}{=} \PY{n}{t}\PY{p}{[}\PY{n}{k}\PY{p}{]}\PY{p}{,} \PY{n}{t}\PY{p}{[}\PY{n}{k} \PY{o}{+} \PY{l+m+mi}{1}\PY{p}{]}
                         \PY{n}{continuer} \PY{o}{=} \PY{k+kc}{True}
                 \PY{n}{bulle} \PY{o}{+}\PY{o}{=} \PY{l+m+mi}{1}
\end{Verbatim}


    \subsubsection{Test de correction du tri par
bulles}\label{test-de-correction-du-tri-par-bulles}

    \begin{Verbatim}[commandchars=\\\{\}]
{\color{incolor}In [{\color{incolor}18}]:} \PY{k}{for} \PY{n}{tri} \PY{o+ow}{in} \PY{p}{[}\PY{n}{tri\PYZus{}bulles}\PY{p}{,} \PY{n}{tri\PYZus{}bulles2}\PY{p}{]}\PY{p}{:}
             \PY{n+nb}{print}\PY{p}{(}\PY{n}{test\PYZus{}tri}\PY{p}{(}\PY{n}{tri}\PY{p}{,} \PY{n}{BENCH1}\PY{p}{)}\PY{p}{)}
             \PY{n+nb}{print}\PY{p}{(}\PY{n}{test\PYZus{}tri}\PY{p}{(}\PY{n}{tri}\PY{p}{,} \PY{n}{BENCH2}\PY{p}{)}\PY{p}{)}
\end{Verbatim}


    \begin{Verbatim}[commandchars=\\\{\}]
[True, True, True]
[True, True, True]
[True, True, True]
[True, True, True]

    \end{Verbatim}

    \subsubsection{Complexité du tri par
bulles}\label{complexituxe9-du-tri-par-bulles}

    \begin{enumerate}
\def\labelenumi{\arabic{enumi}.}
\item
  Tri par bulles de la liste {[}72, 39, 29, 59{]} :

  \begin{itemize}
  \tightlist
  \item
    Etape 1 {[}39,72,29,59{]}
  \item
    Etape 2 {[}39,29,72,59{]}
  \item
    Etape 3 {[}39,29,59,72{]}
  \item
    Etape 4 {[}29,39,59,72{]}
  \end{itemize}
\item
  Implémentation en Python du Tri par bulles, coir ci-dessous plusieurs
  versions.
\item
  Selon l'implémentation \texttt{tri\_bulles} ou \texttt{tri\_bulles2}
  (voir ci-dessus) la complexité du tri par bulles est
  \textbf{quadratique} (comme pour le tri par sélection) ou
  \textbf{linéaire} (comme pour le tri par insertion) sur \emph{les
  listes déjà triées}.
\item
  Si on applique le tri par bulles à une liste triée dans l'ordre
  décroissant comme \([n,n-1,n-2,..,1]\), quelle que soit
  l'implémentation, on a besoin de \(i-1\) permutations (comparaison +
  échange) pour \(i\) allant de \(n\) à \(1\) soit un total de
  \((n-1)+(n-2)+\cdots+1=\frac{n(n-1)}{2}\) permutations. La complexité
  du tri par bulles est donc \textbf{quadratique} sur les listes triées
  dans l'ordre inverse.
\end{enumerate}

    \begin{Verbatim}[commandchars=\\\{\}]
{\color{incolor}In [{\color{incolor}45}]:} \PY{n}{liste\PYZus{}tris} \PY{o}{=} \PY{p}{[}\PY{n}{tri\PYZus{}bulles}\PY{p}{,} \PY{n}{tri\PYZus{}bulles2}\PY{p}{]}
         \PY{n}{liste\PYZus{}taille} \PY{o}{=} \PY{n+nb}{list}\PY{p}{(}\PY{n+nb}{range}\PY{p}{(}\PY{l+m+mi}{10}\PY{p}{,} \PY{l+m+mi}{1001}\PY{p}{,} \PY{l+m+mi}{10}\PY{p}{)}\PY{p}{)}
         \PY{n}{liste\PYZus{}rapport\PYZus{}carre} \PY{o}{=} \PY{n}{np}\PY{o}{.}\PY{n}{array}\PY{p}{(}\PY{p}{[}\PY{p}{[}\PY{l+m+mi}{0} \PY{k}{for} \PY{n}{\PYZus{}} \PY{o+ow}{in} \PY{n+nb}{range}\PY{p}{(}\PY{n+nb}{len}\PY{p}{(}\PY{n}{liste\PYZus{}taille}\PY{p}{)}\PY{p}{)}\PY{p}{]} \PY{k}{for} \PY{n}{\PYZus{}} \PY{o+ow}{in} \PY{n+nb}{range}\PY{p}{(}\PY{n+nb}{len}\PY{p}{(}\PY{n}{liste\PYZus{}tris}\PY{p}{)}\PY{p}{)}\PY{p}{]}\PY{p}{,} \PY{n}{dtype}\PY{o}{=}\PY{l+s+s1}{\PYZsq{}}\PY{l+s+s1}{float}\PY{l+s+s1}{\PYZsq{}}\PY{p}{)}
         \PY{k}{for} \PY{n}{i}\PY{p}{,} \PY{n}{taille} \PY{o+ow}{in} \PY{n+nb}{enumerate}\PY{p}{(}\PY{n}{liste\PYZus{}taille}\PY{p}{)}\PY{p}{:}
             \PY{n}{liste} \PY{o}{=} \PY{p}{[}\PY{n}{randint}\PY{p}{(}\PY{l+m+mi}{0}\PY{p}{,} \PY{n}{taille}\PY{p}{)} \PY{k}{for} \PY{n}{\PYZus{}} \PY{o+ow}{in} \PY{n+nb}{range}\PY{p}{(}\PY{n}{taille}\PY{p}{)}\PY{p}{]}
             \PY{k}{for} \PY{n}{j}\PY{p}{,} \PY{n}{tri} \PY{o+ow}{in} \PY{n+nb}{enumerate}\PY{p}{(}\PY{n}{liste\PYZus{}tris}\PY{p}{)}\PY{p}{:}             
                 \PY{n}{liste\PYZus{}rapport\PYZus{}carre}\PY{p}{[}\PY{n}{j}\PY{p}{]}\PY{p}{[}\PY{n}{i}\PY{p}{]} \PY{o}{=} \PY{n}{timetest}\PY{p}{(}\PY{n}{tri}\PY{p}{)}\PY{p}{(}\PY{n}{liste}\PY{p}{[}\PY{p}{:}\PY{p}{]}\PY{p}{)} \PY{o}{/} \PY{p}{(}\PY{n}{liste\PYZus{}taille}\PY{p}{[}\PY{n}{i}\PY{p}{]}\PY{p}{)} \PY{o}{*}\PY{o}{*} \PY{l+m+mi}{2}        
         \PY{k}{for} \PY{n}{k}\PY{p}{,} \PY{n}{rapport\PYZus{}carre} \PY{o+ow}{in} \PY{n+nb}{enumerate}\PY{p}{(}\PY{n}{liste\PYZus{}rapport\PYZus{}carre}\PY{p}{)}\PY{p}{:}
             \PY{n}{plt}\PY{o}{.}\PY{n}{plot}\PY{p}{(}\PY{n}{liste\PYZus{}taille}\PY{p}{,} \PY{n}{rapport\PYZus{}carre} \PY{o}{/} \PY{n}{rapport\PYZus{}carre}\PY{o}{.}\PY{n}{mean}\PY{p}{(}\PY{p}{)}\PY{p}{,} \PY{n}{label}\PY{o}{=}\PY{n}{liste\PYZus{}tris}\PY{p}{[}\PY{n}{k}\PY{p}{]}\PY{o}{.}\PY{n+nv+vm}{\PYZus{}\PYZus{}name\PYZus{}\PYZus{}}\PY{p}{,} \PY{n}{marker}\PY{o}{=}\PY{l+s+s1}{\PYZsq{}}\PY{l+s+s1}{o}\PY{l+s+s1}{\PYZsq{}}\PY{p}{)}
         \PY{n}{plt}\PY{o}{.}\PY{n}{title}\PY{p}{(}\PY{l+s+sa}{r}\PY{l+s+s2}{\PYZdq{}}\PY{l+s+s2}{Temps / taille ** 2 normalisé par la moyenne}\PY{l+s+s2}{\PYZdq{}}\PY{p}{)}
         \PY{n}{plt}\PY{o}{.}\PY{n}{legend}\PY{p}{(}\PY{p}{)}
         \PY{n}{plt}\PY{o}{.}\PY{n}{savefig}\PY{p}{(}\PY{l+s+s1}{\PYZsq{}}\PY{l+s+s1}{complexite\PYZhy{}quadratique\PYZhy{}tri\PYZus{}bulles.png}\PY{l+s+s1}{\PYZsq{}}\PY{p}{)}
\end{Verbatim}


    \begin{center}
    \adjustimage{max size={0.9\linewidth}{0.9\paperheight}}{output_26_0.png}
    \end{center}
    { \hspace*{\fill} \\}
    
    \subsection{Tri par insertion}\label{tri-par-insertion}

    \begin{enumerate}
\def\labelenumi{\arabic{enumi}.}
\item
  Descriptions du tri par insertion :

  \begin{itemize}
  \tightlist
  \item
    \href{http://fr.wikipedia.org/wiki/Tri_par_insertion}{Article de
    Wikipedia}
  \item
    \href{http://www.sorting-algorithms.com/insertion-sort}{Le site
    sorting algorithms}
  \end{itemize}
\item
  Implémentation du tri par insertion en Python (voir ci-dessous)
\item
  Contrairement au tri par sélection, si l'on interrompt l'exécution de
  l'algorithme du tri par sélection après \(k\) étapes après \(k\)
  étapes, la sous-liste des \(k\) premiers éléments déjà triés n'est pas
  celle des \(k\) plus petits élements de la liste triée finale.
\end{enumerate}

    \subsubsection{Implémentations du tri par
insertion}\label{impluxe9mentations-du-tri-par-insertion}

    \begin{Verbatim}[commandchars=\\\{\}]
{\color{incolor}In [{\color{incolor}19}]:} \PY{k}{def} \PY{n+nf}{insertion}\PY{p}{(}\PY{n}{t}\PY{p}{,} \PY{n}{k}\PY{p}{)}\PY{p}{:}   
             \PY{l+s+sd}{\PYZdq{}\PYZdq{}\PYZdq{}A partir de l\PYZsq{}indice k, insère l\PYZsq{}élément t[k] de t }
         \PY{l+s+sd}{    à sa place dans t[:k] triée dans l\PYZsq{}ordre croissant\PYZdq{}\PYZdq{}\PYZdq{}}
             \PY{n}{i} \PY{o}{=} \PY{n}{k} \PY{o}{\PYZhy{}} \PY{l+m+mi}{1}
             \PY{n}{element} \PY{o}{=} \PY{n}{t}\PY{p}{[}\PY{n}{k}\PY{p}{]}
             \PY{k}{while} \PY{n}{i} \PY{o}{\PYZgt{}}\PY{o}{=} \PY{l+m+mi}{0} \PY{o+ow}{and} \PY{n}{t}\PY{p}{[}\PY{n}{i}\PY{p}{]} \PY{o}{\PYZgt{}} \PY{n}{element}\PY{p}{:}
                 \PY{n}{t}\PY{p}{[}\PY{n}{i} \PY{o}{+} \PY{l+m+mi}{1}\PY{p}{]} \PY{o}{=} \PY{n}{t}\PY{p}{[}\PY{n}{i}\PY{p}{]}
                 \PY{n}{i} \PY{o}{=} \PY{n}{i} \PY{o}{\PYZhy{}} \PY{l+m+mi}{1}
             \PY{n}{t}\PY{p}{[}\PY{n}{i}\PY{p}{]} \PY{o}{=} \PY{n}{element}
             
         \PY{k}{def} \PY{n+nf}{tri\PYZus{}insertion}\PY{p}{(}\PY{n}{t}\PY{p}{)}\PY{p}{:}
             \PY{n}{n} \PY{o}{=} \PY{n+nb}{len}\PY{p}{(}\PY{n}{t}\PY{p}{)}
             \PY{k}{for} \PY{n}{k} \PY{o+ow}{in} \PY{n+nb}{range}\PY{p}{(}\PY{l+m+mi}{1}\PY{p}{,} \PY{n}{n}\PY{p}{)}\PY{p}{:}
                 \PY{n}{insertion}\PY{p}{(}\PY{n}{t}\PY{p}{,} \PY{n}{k}\PY{p}{)}
\end{Verbatim}


    \subsubsection{Test de correction du tri par
insertion}\label{test-de-correction-du-tri-par-insertion}

    \begin{Verbatim}[commandchars=\\\{\}]
{\color{incolor}In [{\color{incolor}20}]:} \PY{k}{for} \PY{n}{tri} \PY{o+ow}{in} \PY{p}{[}\PY{n}{tri\PYZus{}insertion}\PY{p}{]}\PY{p}{:}
             \PY{n+nb}{print}\PY{p}{(}\PY{n}{test\PYZus{}tri}\PY{p}{(}\PY{n}{tri}\PY{p}{,} \PY{n}{BENCH1}\PY{p}{)}\PY{p}{)}
             \PY{n+nb}{print}\PY{p}{(}\PY{n}{test\PYZus{}tri}\PY{p}{(}\PY{n}{tri}\PY{p}{,} \PY{n}{BENCH2}\PY{p}{)}\PY{p}{)}
\end{Verbatim}


    \begin{Verbatim}[commandchars=\\\{\}]
[True, True, True]
[True, True, True]

    \end{Verbatim}

    \subsubsection{Complexité du tri par
insertion}\label{complexituxe9-du-tri-par-insertion}

    Analyse de complexité du tri par insertion :

\begin{itemize}
\tightlist
\item
  Dans \textbf{le meilleur des cas} le tableau (ou liste) est déjà trié,
  on effectue alors une seule comparaison à chaque insertion soit
  \(n-1\) comparaisons au total
\item
  \textbf{En moyenne} (liste d'entiers aléatoires) le nombre de
  comparaisons est de l'ordre de \(n-1+\frac{n(n-1)}{4}\)
\item
  \textbf{Dans le pire des cas} le tableau est trié dans l'ordre inverse
  on effectue \(i\) comparaisons lors de l'insertion d'indice \(i\) soit
  \(1+2+\cdots+(n-2)+(n-1)+\cdots+1=\frac{n(n-1)}{2}\) comparaisons.
\end{itemize}

\textbf{La complexité du tri par insertion } est donc
\textbf{quadratique} dans le cas moyen et le pire des cas et
\textbf{linéaire} dans le meilleur des cas (liste déjà triée). Cette
bonne propriété que ne possèdent pas les tris par sélection ou par
bulles, font du tri par insertion un tri efficace lorsqu'il y a peu de
comparaisons ou que la liste est presque déjà triée.

    \begin{Verbatim}[commandchars=\\\{\}]
{\color{incolor}In [{\color{incolor}44}]:} \PY{n}{liste\PYZus{}tris} \PY{o}{=} \PY{p}{[}\PY{n}{tri\PYZus{}insertion}\PY{p}{]}
         \PY{n}{liste\PYZus{}taille} \PY{o}{=} \PY{n+nb}{list}\PY{p}{(}\PY{n+nb}{range}\PY{p}{(}\PY{l+m+mi}{10}\PY{p}{,} \PY{l+m+mi}{1001}\PY{p}{,} \PY{l+m+mi}{10}\PY{p}{)}\PY{p}{)}
         \PY{n}{liste\PYZus{}rapport\PYZus{}carre} \PY{o}{=} \PY{n}{np}\PY{o}{.}\PY{n}{array}\PY{p}{(}\PY{p}{[}\PY{p}{[}\PY{l+m+mi}{0} \PY{k}{for} \PY{n}{\PYZus{}} \PY{o+ow}{in} \PY{n+nb}{range}\PY{p}{(}\PY{n+nb}{len}\PY{p}{(}\PY{n}{liste\PYZus{}taille}\PY{p}{)}\PY{p}{)}\PY{p}{]} \PY{k}{for} \PY{n}{\PYZus{}} \PY{o+ow}{in} \PY{n+nb}{range}\PY{p}{(}\PY{n+nb}{len}\PY{p}{(}\PY{n}{liste\PYZus{}tris}\PY{p}{)}\PY{p}{)}\PY{p}{]}\PY{p}{,} \PY{n}{dtype}\PY{o}{=}\PY{l+s+s1}{\PYZsq{}}\PY{l+s+s1}{float}\PY{l+s+s1}{\PYZsq{}}\PY{p}{)}
         \PY{k}{for} \PY{n}{i}\PY{p}{,} \PY{n}{taille} \PY{o+ow}{in} \PY{n+nb}{enumerate}\PY{p}{(}\PY{n}{liste\PYZus{}taille}\PY{p}{)}\PY{p}{:}
             \PY{n}{liste} \PY{o}{=} \PY{p}{[}\PY{n}{randint}\PY{p}{(}\PY{l+m+mi}{0}\PY{p}{,} \PY{n}{taille}\PY{p}{)} \PY{k}{for} \PY{n}{\PYZus{}} \PY{o+ow}{in} \PY{n+nb}{range}\PY{p}{(}\PY{n}{taille}\PY{p}{)}\PY{p}{]}
             \PY{k}{for} \PY{n}{j}\PY{p}{,} \PY{n}{tri} \PY{o+ow}{in} \PY{n+nb}{enumerate}\PY{p}{(}\PY{n}{liste\PYZus{}tris}\PY{p}{)}\PY{p}{:}             
                 \PY{n}{liste\PYZus{}rapport\PYZus{}carre}\PY{p}{[}\PY{n}{j}\PY{p}{]}\PY{p}{[}\PY{n}{i}\PY{p}{]} \PY{o}{=} \PY{n}{timetest}\PY{p}{(}\PY{n}{tri}\PY{p}{)}\PY{p}{(}\PY{n}{liste}\PY{p}{[}\PY{p}{:}\PY{p}{]}\PY{p}{)} \PY{o}{/} \PY{p}{(}\PY{n}{liste\PYZus{}taille}\PY{p}{[}\PY{n}{i}\PY{p}{]}\PY{p}{)} \PY{o}{*}\PY{o}{*} \PY{l+m+mi}{2}        
         \PY{k}{for} \PY{n}{k}\PY{p}{,} \PY{n}{rapport\PYZus{}carre} \PY{o+ow}{in} \PY{n+nb}{enumerate}\PY{p}{(}\PY{n}{liste\PYZus{}rapport\PYZus{}carre}\PY{p}{)}\PY{p}{:}
             \PY{n}{plt}\PY{o}{.}\PY{n}{plot}\PY{p}{(}\PY{n}{liste\PYZus{}taille}\PY{p}{,} \PY{n}{rapport\PYZus{}carre} \PY{o}{/} \PY{n}{rapport\PYZus{}carre}\PY{o}{.}\PY{n}{mean}\PY{p}{(}\PY{p}{)}\PY{p}{,} \PY{n}{label}\PY{o}{=}\PY{n}{liste\PYZus{}tris}\PY{p}{[}\PY{n}{k}\PY{p}{]}\PY{o}{.}\PY{n+nv+vm}{\PYZus{}\PYZus{}name\PYZus{}\PYZus{}}\PY{p}{,} \PY{n}{marker}\PY{o}{=}\PY{l+s+s1}{\PYZsq{}}\PY{l+s+s1}{o}\PY{l+s+s1}{\PYZsq{}}\PY{p}{)}
         \PY{n}{plt}\PY{o}{.}\PY{n}{title}\PY{p}{(}\PY{l+s+sa}{r}\PY{l+s+s2}{\PYZdq{}}\PY{l+s+s2}{Temps / taille ** 2 normalisé par la moyenne}\PY{l+s+s2}{\PYZdq{}}\PY{p}{)}
         \PY{n}{plt}\PY{o}{.}\PY{n}{legend}\PY{p}{(}\PY{p}{)}
         \PY{n}{plt}\PY{o}{.}\PY{n}{savefig}\PY{p}{(}\PY{l+s+s1}{\PYZsq{}}\PY{l+s+s1}{complexite\PYZhy{}quadratique\PYZhy{}tri\PYZus{}insertion.png}\PY{l+s+s1}{\PYZsq{}}\PY{p}{)}
\end{Verbatim}


    \begin{center}
    \adjustimage{max size={0.9\linewidth}{0.9\paperheight}}{output_35_0.png}
    \end{center}
    { \hspace*{\fill} \\}
    
    \subsection{Tri fusion}\label{tri-fusion}

    Descriptions du tri par fusion :

\begin{itemize}
\tightlist
\item
  \href{http://fr.wikipedia.org/wiki/Tri_fusion}{Article de Wikipedia}
\item
  \href{http://www.sorting-algorithms.com/merge-sort}{Le site sorting
  algorithms}
\item
  \href{https://interstices.info/jcms/c_6973/les-algorithmes-de-tri}{Article
  du site Interstices}
\end{itemize}

    \subsubsection{Implémentation du tri
fusion}\label{impluxe9mentation-du-tri-fusion}

    \begin{Verbatim}[commandchars=\\\{\}]
{\color{incolor}In [{\color{incolor}48}]:} \PY{k}{def} \PY{n+nf}{fusion}\PY{p}{(}\PY{n}{liste}\PY{p}{,} \PY{n}{p}\PY{p}{,} \PY{n}{q}\PY{p}{,} \PY{n}{r}\PY{p}{)}\PY{p}{:}
             \PY{n}{tampon} \PY{o}{=} \PY{p}{[}\PY{p}{]}
             \PY{n}{i} \PY{o}{=} \PY{n}{p}
             \PY{n}{j} \PY{o}{=} \PY{n}{q}    
             \PY{k}{for} \PY{n}{k} \PY{o+ow}{in} \PY{n+nb}{range}\PY{p}{(}\PY{n}{r} \PY{o}{\PYZhy{}} \PY{n}{p}\PY{p}{)}\PY{p}{:}            \PY{c+c1}{\PYZsh{}fusion des sous\PYZhy{}listes triées }
                 \PY{k}{if} \PY{n}{j} \PY{o}{==} \PY{n}{r} \PY{o+ow}{or} \PY{p}{(}\PY{n}{j} \PY{o}{\PYZlt{}} \PY{n}{r} \PY{o+ow}{and} \PY{n}{i} \PY{o}{\PYZlt{}} \PY{n}{q} \PY{o+ow}{and} \PY{n}{liste}\PY{p}{[}\PY{n}{i}\PY{p}{]} \PY{o}{\PYZlt{}}\PY{o}{=} \PY{n}{liste}\PY{p}{[}\PY{n}{j}\PY{p}{]}\PY{p}{)}\PY{p}{:}
                     \PY{n}{tampon}\PY{o}{.}\PY{n}{append}\PY{p}{(}\PY{n}{liste}\PY{p}{[}\PY{n}{i}\PY{p}{]}\PY{p}{)}
                 \PY{k}{else}\PY{p}{:}
                     \PY{n}{tampon}\PY{o}{.}\PY{n}{append}\PY{p}{(}\PY{n}{liste}\PY{p}{[}\PY{n}{j}\PY{p}{]}\PY{p}{)}
             \PY{k}{for} \PY{n}{k} \PY{o+ow}{in} \PY{n+nb}{range}\PY{p}{(}\PY{n}{r} \PY{o}{\PYZhy{}} \PY{n}{p}\PY{p}{)}\PY{p}{:}            \PY{c+c1}{\PYZsh{}recopie de tampon  dans liste[p:r]}
                 \PY{n}{liste}\PY{p}{[}\PY{n}{p} \PY{o}{+} \PY{n}{k}\PY{p}{]} \PY{o}{=} \PY{n}{tampon}\PY{p}{[}\PY{n}{k}\PY{p}{]}
         
         \PY{k}{def} \PY{n+nf}{tri\PYZus{}fusion}\PY{p}{(}\PY{n}{liste}\PY{p}{,} \PY{n}{p}\PY{p}{,} \PY{n}{r}\PY{p}{)}\PY{p}{:}
             \PY{k}{if} \PY{n}{p} \PY{o}{+} \PY{l+m+mi}{1} \PY{o}{\PYZlt{}} \PY{n}{r}\PY{p}{:}
                 \PY{n}{q} \PY{o}{=} \PY{p}{(}\PY{n}{p} \PY{o}{+} \PY{n}{r}\PY{p}{)} \PY{o}{/}\PY{o}{/} \PY{l+m+mi}{2}
                 \PY{n}{tri\PYZus{}fusion}\PY{p}{(}\PY{n}{liste}\PY{p}{,} \PY{n}{p}\PY{p}{,} \PY{n}{q}\PY{p}{)}
                 \PY{n}{tri\PYZus{}fusion}\PY{p}{(}\PY{n}{liste}\PY{p}{,} \PY{n}{q}\PY{p}{,} \PY{n}{r}\PY{p}{)}
                 \PY{n}{fusion}\PY{p}{(}\PY{n}{liste}\PY{p}{,} \PY{n}{p}\PY{p}{,} \PY{n}{q}\PY{p}{,} \PY{n}{r}\PY{p}{)}        
\end{Verbatim}


    \subsubsection{Test de correction du tri
fusion}\label{test-de-correction-du-tri-fusion}

    \begin{Verbatim}[commandchars=\\\{\}]
{\color{incolor}In [{\color{incolor}50}]:} \PY{k}{for} \PY{n}{tri} \PY{o+ow}{in} \PY{p}{[}\PY{k}{lambda} \PY{n}{liste} \PY{p}{:} \PY{n}{tri\PYZus{}fusion}\PY{p}{(}\PY{n}{liste}\PY{p}{,} \PY{l+m+mi}{0}\PY{p}{,} \PY{n+nb}{len}\PY{p}{(}\PY{n}{liste}\PY{p}{)}\PY{p}{)}\PY{p}{]}\PY{p}{:}
             \PY{n+nb}{print}\PY{p}{(}\PY{n}{test\PYZus{}tri}\PY{p}{(}\PY{n}{tri}\PY{p}{,} \PY{n}{BENCH1}\PY{p}{)}\PY{p}{)}
             \PY{n+nb}{print}\PY{p}{(}\PY{n}{test\PYZus{}tri}\PY{p}{(}\PY{n}{tri}\PY{p}{,} \PY{n}{BENCH2}\PY{p}{)}\PY{p}{)}
\end{Verbatim}


    \begin{Verbatim}[commandchars=\\\{\}]
[True, True, True]
[True, True, True]

    \end{Verbatim}

    \subsubsection{Complexité du tri
fusion}\label{complexituxe9-du-tri-fusion}

    La complexité du \textbf{tri fusion} sur une liste de taille \(n\) est
de l'ordre de grandeur de \(n \ln(n)\) :

\begin{itemize}
\tightlist
\item
  la profondeur de l'arbre d'appels récursifs est de l'ordre de
  \(\frac{\ln(n)}{\ln(2)}\) le logarithme binaire de \(n\)
\item
  au niveau \(k\) on a \(2^{k}\) sous-listes de taille
  \(\frac{n}{2^{k}}\) qu'on fusionne en \(2^{k} \times \frac{1}{2}\)
  appels à la fonction \textbf{fusion} et chaque appel à \textbf{fusion}
  coûte \(2 \times \frac{n}{2^{k}}\) comparaisons ; on a donc
  \(2^{k} \times \frac{1}{2} \times \frac{n}{2^{k}} \times 2 = n\)
  comparaisons par niveau de l'arbre
\item
  le nombre total de comparaisons dans l'arbre est donc bien de l'ordre
  de grandeur de \(\frac{\ln(n)}{\ln(2)} \times n\)
\end{itemize}

    \begin{Verbatim}[commandchars=\\\{\}]
{\color{incolor}In [{\color{incolor}76}]:} \PY{k}{def}  \PY{n+nf}{tri\PYZus{}Fusion}\PY{p}{(}\PY{n}{liste}\PY{p}{)}\PY{p}{:} \PY{k}{return} \PY{n}{tri\PYZus{}fusion}\PY{p}{(}\PY{n}{liste}\PY{p}{,} \PY{l+m+mi}{0}\PY{p}{,} \PY{n+nb}{len}\PY{p}{(}\PY{n}{liste}\PY{p}{)}\PY{p}{)}
         \PY{n}{liste\PYZus{}tris} \PY{o}{=} \PY{p}{[}\PY{n}{tri\PYZus{}Fusion}\PY{p}{]}
         \PY{n}{liste\PYZus{}taille} \PY{o}{=} \PY{n+nb}{list}\PY{p}{(}\PY{n+nb}{range}\PY{p}{(}\PY{l+m+mi}{100}\PY{p}{,} \PY{l+m+mi}{10001}\PY{p}{,} \PY{l+m+mi}{100}\PY{p}{)}\PY{p}{)}
         \PY{n}{liste\PYZus{}rapport\PYZus{}taille} \PY{o}{=} \PY{n}{np}\PY{o}{.}\PY{n}{array}\PY{p}{(}\PY{p}{[}\PY{p}{[}\PY{l+m+mi}{0} \PY{k}{for} \PY{n}{\PYZus{}} \PY{o+ow}{in} \PY{n+nb}{range}\PY{p}{(}\PY{n+nb}{len}\PY{p}{(}\PY{n}{liste\PYZus{}taille}\PY{p}{)}\PY{p}{)}\PY{p}{]} \PY{k}{for} \PY{n}{\PYZus{}} \PY{o+ow}{in} \PY{n+nb}{range}\PY{p}{(}\PY{n+nb}{len}\PY{p}{(}\PY{n}{liste\PYZus{}tris}\PY{p}{)}\PY{p}{)}\PY{p}{]}\PY{p}{,} \PY{n}{dtype}\PY{o}{=}\PY{l+s+s1}{\PYZsq{}}\PY{l+s+s1}{float}\PY{l+s+s1}{\PYZsq{}}\PY{p}{)}
         \PY{k}{for} \PY{n}{i}\PY{p}{,} \PY{n}{taille} \PY{o+ow}{in} \PY{n+nb}{enumerate}\PY{p}{(}\PY{n}{liste\PYZus{}taille}\PY{p}{)}\PY{p}{:}
             \PY{n}{liste} \PY{o}{=} \PY{p}{[}\PY{n}{randint}\PY{p}{(}\PY{l+m+mi}{0}\PY{p}{,} \PY{n}{taille}\PY{p}{)} \PY{k}{for} \PY{n}{\PYZus{}} \PY{o+ow}{in} \PY{n+nb}{range}\PY{p}{(}\PY{n}{taille}\PY{p}{)}\PY{p}{]}
             \PY{k}{for} \PY{n}{j}\PY{p}{,} \PY{n}{tri} \PY{o+ow}{in} \PY{n+nb}{enumerate}\PY{p}{(}\PY{n}{liste\PYZus{}tris}\PY{p}{)}\PY{p}{:}             
                 \PY{n}{liste\PYZus{}rapport\PYZus{}taille}\PY{p}{[}\PY{n}{j}\PY{p}{]}\PY{p}{[}\PY{n}{i}\PY{p}{]} \PY{o}{=} \PY{n}{timetest}\PY{p}{(}\PY{n}{tri}\PY{p}{)}\PY{p}{(}\PY{n}{liste}\PY{p}{[}\PY{p}{:}\PY{p}{]}\PY{p}{)} \PY{o}{/} \PY{p}{(}\PY{n}{liste\PYZus{}taille}\PY{p}{[}\PY{n}{i}\PY{p}{]} \PY{o}{*} \PY{n}{np}\PY{o}{.}\PY{n}{log}\PY{p}{(}\PY{n}{liste\PYZus{}taille}\PY{p}{[}\PY{n}{i}\PY{p}{]}\PY{p}{)}\PY{p}{)}        
         \PY{k}{for} \PY{n}{k}\PY{p}{,} \PY{n}{rapport\PYZus{}taille}  \PY{o+ow}{in} \PY{n+nb}{enumerate}\PY{p}{(}\PY{n}{liste\PYZus{}rapport\PYZus{}taille}\PY{p}{)}\PY{p}{:}
             \PY{n}{plt}\PY{o}{.}\PY{n}{plot}\PY{p}{(}\PY{n}{liste\PYZus{}taille}\PY{p}{,} \PY{n}{rapport\PYZus{}taille} \PY{o}{/} \PY{n}{rapport\PYZus{}taille}\PY{o}{.}\PY{n}{mean}\PY{p}{(}\PY{p}{)}\PY{p}{,} \PY{n}{label}\PY{o}{=}\PY{n}{liste\PYZus{}tris}\PY{p}{[}\PY{n}{k}\PY{p}{]}\PY{o}{.}\PY{n+nv+vm}{\PYZus{}\PYZus{}name\PYZus{}\PYZus{}}\PY{p}{,} \PY{n}{marker}\PY{o}{=}\PY{l+s+s1}{\PYZsq{}}\PY{l+s+s1}{o}\PY{l+s+s1}{\PYZsq{}}\PY{p}{)}
         \PY{n}{plt}\PY{o}{.}\PY{n}{title}\PY{p}{(}\PY{l+s+sa}{r}\PY{l+s+s2}{\PYZdq{}}\PY{l+s+s2}{Temps / (taille * ln(taille)) normalisé par la moyenne}\PY{l+s+s2}{\PYZdq{}}\PY{p}{)}
         \PY{n}{plt}\PY{o}{.}\PY{n}{legend}\PY{p}{(}\PY{p}{)}
         \PY{n}{plt}\PY{o}{.}\PY{n}{savefig}\PY{p}{(}\PY{l+s+s1}{\PYZsq{}}\PY{l+s+s1}{complexite\PYZhy{}nln(n)\PYZhy{}tri\PYZus{}fusion.png}\PY{l+s+s1}{\PYZsq{}}\PY{p}{)}
\end{Verbatim}


    \begin{center}
    \adjustimage{max size={0.9\linewidth}{0.9\paperheight}}{output_44_0.png}
    \end{center}
    { \hspace*{\fill} \\}
    
    \subsection{Comparaison des tris}\label{comparaison-des-tris}

    \subsubsection{Les tris en lice}\label{les-tris-en-lice}

    \begin{Verbatim}[commandchars=\\\{\}]
{\color{incolor}In [{\color{incolor}66}]:} \PY{k}{def}  \PY{n+nf}{tri\PYZus{}Fusion}\PY{p}{(}\PY{n}{liste}\PY{p}{)}\PY{p}{:} \PY{k}{return} \PY{n}{tri\PYZus{}fusion}\PY{p}{(}\PY{n}{liste}\PY{p}{,} \PY{l+m+mi}{0}\PY{p}{,} \PY{n+nb}{len}\PY{p}{(}\PY{n}{liste}\PY{p}{)}\PY{p}{)}
         \PY{n}{liste\PYZus{}tris} \PY{o}{=} \PY{p}{[}\PY{n}{tri\PYZus{}selection}\PY{p}{,} \PY{n}{tri\PYZus{}bulles2}\PY{p}{,} \PY{n}{tri\PYZus{}insertion}\PY{p}{,} \PY{n}{tri\PYZus{}Fusion}\PY{p}{]}
\end{Verbatim}


    \subsubsection{Comparaison des tris sur des listes
aléatoires}\label{comparaison-des-tris-sur-des-listes-aluxe9atoires}

    \begin{Verbatim}[commandchars=\\\{\}]
{\color{incolor}In [{\color{incolor}71}]:} \PY{n}{liste\PYZus{}taille} \PY{o}{=} \PY{n+nb}{list}\PY{p}{(}\PY{n+nb}{range}\PY{p}{(}\PY{l+m+mi}{0}\PY{p}{,} \PY{l+m+mi}{100}\PY{p}{,} \PY{l+m+mi}{10}\PY{p}{)}\PY{p}{)} \PY{o}{+} \PY{n+nb}{list}\PY{p}{(}\PY{n+nb}{range}\PY{p}{(}\PY{l+m+mi}{100}\PY{p}{,} \PY{l+m+mi}{1000}\PY{p}{,} \PY{l+m+mi}{100}\PY{p}{)}\PY{p}{)} \PY{o}{+} \PY{n+nb}{list}\PY{p}{(}\PY{n+nb}{range}\PY{p}{(}\PY{l+m+mi}{1000}\PY{p}{,} \PY{l+m+mi}{5001}\PY{p}{,} \PY{l+m+mi}{1000}\PY{p}{)}\PY{p}{)}
         \PY{n}{liste\PYZus{}temps} \PY{o}{=} \PY{p}{[}\PY{p}{[}\PY{l+m+mi}{0} \PY{k}{for} \PY{n}{\PYZus{}} \PY{o+ow}{in} \PY{n+nb}{range}\PY{p}{(}\PY{n+nb}{len}\PY{p}{(}\PY{n}{liste\PYZus{}taille}\PY{p}{)}\PY{p}{)}\PY{p}{]} \PY{k}{for} \PY{n}{\PYZus{}} \PY{o+ow}{in} \PY{n+nb}{range}\PY{p}{(}\PY{n+nb}{len}\PY{p}{(}\PY{n}{liste\PYZus{}tris}\PY{p}{)}\PY{p}{)}\PY{p}{]}
         \PY{k}{for} \PY{n}{i}\PY{p}{,} \PY{n}{taille} \PY{o+ow}{in} \PY{n+nb}{enumerate}\PY{p}{(}\PY{n}{liste\PYZus{}taille}\PY{p}{)}\PY{p}{:}
             \PY{n}{liste} \PY{o}{=} \PY{p}{[}\PY{n}{randint}\PY{p}{(}\PY{l+m+mi}{0}\PY{p}{,} \PY{n}{taille}\PY{p}{)} \PY{k}{for} \PY{n}{\PYZus{}} \PY{o+ow}{in} \PY{n+nb}{range}\PY{p}{(}\PY{n}{taille}\PY{p}{)}\PY{p}{]}
             \PY{k}{for} \PY{n}{j}\PY{p}{,} \PY{n}{tri} \PY{o+ow}{in} \PY{n+nb}{enumerate}\PY{p}{(}\PY{n}{liste\PYZus{}tris}\PY{p}{)}\PY{p}{:}
                 \PY{n}{liste\PYZus{}temps}\PY{p}{[}\PY{n}{j}\PY{p}{]}\PY{p}{[}\PY{n}{i}\PY{p}{]} \PY{o}{=} \PY{n}{timetest}\PY{p}{(}\PY{n}{tri}\PY{p}{)}\PY{p}{(}\PY{n}{liste}\PY{p}{[}\PY{p}{:}\PY{p}{]}\PY{p}{)}
         \PY{k}{for} \PY{n}{k}\PY{p}{,} \PY{n}{temps} \PY{o+ow}{in} \PY{n+nb}{enumerate}\PY{p}{(}\PY{n}{liste\PYZus{}temps}\PY{p}{)}\PY{p}{:}
             \PY{n}{plt}\PY{o}{.}\PY{n}{loglog}\PY{p}{(}\PY{n}{liste\PYZus{}taille}\PY{p}{,} \PY{n}{temps}\PY{p}{,} \PY{n}{label}\PY{o}{=}\PY{n}{liste\PYZus{}tris}\PY{p}{[}\PY{n}{k}\PY{p}{]}\PY{o}{.}\PY{n+nv+vm}{\PYZus{}\PYZus{}name\PYZus{}\PYZus{}}\PY{p}{,} \PY{n}{marker}\PY{o}{=}\PY{l+s+s1}{\PYZsq{}}\PY{l+s+s1}{o}\PY{l+s+s1}{\PYZsq{}}\PY{p}{)}
         \PY{n}{plt}\PY{o}{.}\PY{n}{legend}\PY{p}{(}\PY{p}{)}
         \PY{n}{plt}\PY{o}{.}\PY{n}{savefig}\PY{p}{(}\PY{l+s+s1}{\PYZsq{}}\PY{l+s+s1}{test\PYZhy{}tris\PYZhy{}listes\PYZhy{}aleatoires.png}\PY{l+s+s1}{\PYZsq{}}\PY{p}{)}
\end{Verbatim}


    \begin{center}
    \adjustimage{max size={0.9\linewidth}{0.9\paperheight}}{output_49_0.png}
    \end{center}
    { \hspace*{\fill} \\}
    
    \subsubsection{Comparaison des tris sur des listes triées dans l'ordre
inverse}\label{comparaison-des-tris-sur-des-listes-triuxe9es-dans-lordre-inverse}

    \begin{Verbatim}[commandchars=\\\{\}]
{\color{incolor}In [{\color{incolor}70}]:} \PY{n}{liste\PYZus{}taille} \PY{o}{=} \PY{n+nb}{list}\PY{p}{(}\PY{n+nb}{range}\PY{p}{(}\PY{l+m+mi}{0}\PY{p}{,} \PY{l+m+mi}{100}\PY{p}{,} \PY{l+m+mi}{10}\PY{p}{)}\PY{p}{)} \PY{o}{+} \PY{n+nb}{list}\PY{p}{(}\PY{n+nb}{range}\PY{p}{(}\PY{l+m+mi}{100}\PY{p}{,} \PY{l+m+mi}{1000}\PY{p}{,} \PY{l+m+mi}{100}\PY{p}{)}\PY{p}{)} \PY{o}{+} \PY{n+nb}{list}\PY{p}{(}\PY{n+nb}{range}\PY{p}{(}\PY{l+m+mi}{1000}\PY{p}{,} \PY{l+m+mi}{5001}\PY{p}{,} \PY{l+m+mi}{1000}\PY{p}{)}\PY{p}{)}
         \PY{n}{liste\PYZus{}temps} \PY{o}{=} \PY{p}{[}\PY{p}{[}\PY{l+m+mi}{0} \PY{k}{for} \PY{n}{\PYZus{}} \PY{o+ow}{in} \PY{n+nb}{range}\PY{p}{(}\PY{n+nb}{len}\PY{p}{(}\PY{n}{liste\PYZus{}taille}\PY{p}{)}\PY{p}{)}\PY{p}{]} \PY{k}{for} \PY{n}{\PYZus{}} \PY{o+ow}{in} \PY{n+nb}{range}\PY{p}{(}\PY{n+nb}{len}\PY{p}{(}\PY{n}{liste\PYZus{}tris}\PY{p}{)}\PY{p}{)}\PY{p}{]}
         \PY{k}{for} \PY{n}{i}\PY{p}{,} \PY{n}{taille} \PY{o+ow}{in} \PY{n+nb}{enumerate}\PY{p}{(}\PY{n}{liste\PYZus{}taille}\PY{p}{)}\PY{p}{:}
             \PY{n}{liste} \PY{o}{=} \PY{n+nb}{list}\PY{p}{(}\PY{n+nb}{range}\PY{p}{(}\PY{n}{taille}\PY{p}{)}\PY{p}{)}\PY{p}{[}\PY{p}{:}\PY{p}{:}\PY{o}{\PYZhy{}}\PY{l+m+mi}{1}\PY{p}{]}
             \PY{k}{for} \PY{n}{j}\PY{p}{,} \PY{n}{tri} \PY{o+ow}{in} \PY{n+nb}{enumerate}\PY{p}{(}\PY{n}{liste\PYZus{}tris}\PY{p}{)}\PY{p}{:}
                 \PY{n}{liste\PYZus{}temps}\PY{p}{[}\PY{n}{j}\PY{p}{]}\PY{p}{[}\PY{n}{i}\PY{p}{]} \PY{o}{=} \PY{n}{timetest}\PY{p}{(}\PY{n}{tri}\PY{p}{)}\PY{p}{(}\PY{n}{liste}\PY{p}{[}\PY{p}{:}\PY{p}{]}\PY{p}{)}
         \PY{k}{for} \PY{n}{k}\PY{p}{,} \PY{n}{temps} \PY{o+ow}{in} \PY{n+nb}{enumerate}\PY{p}{(}\PY{n}{liste\PYZus{}temps}\PY{p}{)}\PY{p}{:}
             \PY{n}{plt}\PY{o}{.}\PY{n}{loglog}\PY{p}{(}\PY{n}{liste\PYZus{}taille}\PY{p}{,} \PY{n}{temps}\PY{p}{,} \PY{n}{label}\PY{o}{=}\PY{n}{liste\PYZus{}tris}\PY{p}{[}\PY{n}{k}\PY{p}{]}\PY{o}{.}\PY{n+nv+vm}{\PYZus{}\PYZus{}name\PYZus{}\PYZus{}}\PY{p}{,} \PY{n}{marker}\PY{o}{=}\PY{l+s+s1}{\PYZsq{}}\PY{l+s+s1}{o}\PY{l+s+s1}{\PYZsq{}}\PY{p}{)}
         \PY{n}{plt}\PY{o}{.}\PY{n}{legend}\PY{p}{(}\PY{p}{)}
         \PY{n}{plt}\PY{o}{.}\PY{n}{savefig}\PY{p}{(}\PY{l+s+s1}{\PYZsq{}}\PY{l+s+s1}{test\PYZhy{}tris\PYZhy{}listes\PYZhy{}ordre\PYZhy{}inverse.png}\PY{l+s+s1}{\PYZsq{}}\PY{p}{)}
\end{Verbatim}


    \begin{center}
    \adjustimage{max size={0.9\linewidth}{0.9\paperheight}}{output_51_0.png}
    \end{center}
    { \hspace*{\fill} \\}
    
    \subsubsection{Comparaison des tris sur des listes déjà
triées}\label{comparaison-des-tris-sur-des-listes-duxe9juxe0-triuxe9es}

    \begin{Verbatim}[commandchars=\\\{\}]
{\color{incolor}In [{\color{incolor}73}]:} \PY{n}{liste\PYZus{}taille} \PY{o}{=} \PY{n+nb}{list}\PY{p}{(}\PY{n+nb}{range}\PY{p}{(}\PY{l+m+mi}{0}\PY{p}{,} \PY{l+m+mi}{100}\PY{p}{,} \PY{l+m+mi}{10}\PY{p}{)}\PY{p}{)} \PY{o}{+} \PY{n+nb}{list}\PY{p}{(}\PY{n+nb}{range}\PY{p}{(}\PY{l+m+mi}{100}\PY{p}{,} \PY{l+m+mi}{1000}\PY{p}{,} \PY{l+m+mi}{100}\PY{p}{)}\PY{p}{)} \PY{o}{+} \PY{n+nb}{list}\PY{p}{(}\PY{n+nb}{range}\PY{p}{(}\PY{l+m+mi}{1000}\PY{p}{,} \PY{l+m+mi}{5001}\PY{p}{,} \PY{l+m+mi}{1000}\PY{p}{)}\PY{p}{)}
         \PY{n}{liste\PYZus{}temps} \PY{o}{=} \PY{p}{[}\PY{p}{[}\PY{l+m+mi}{0} \PY{k}{for} \PY{n}{\PYZus{}} \PY{o+ow}{in} \PY{n+nb}{range}\PY{p}{(}\PY{n+nb}{len}\PY{p}{(}\PY{n}{liste\PYZus{}taille}\PY{p}{)}\PY{p}{)}\PY{p}{]} \PY{k}{for} \PY{n}{\PYZus{}} \PY{o+ow}{in} \PY{n+nb}{range}\PY{p}{(}\PY{n+nb}{len}\PY{p}{(}\PY{n}{liste\PYZus{}tris}\PY{p}{)}\PY{p}{)}\PY{p}{]}
         \PY{k}{for} \PY{n}{i}\PY{p}{,} \PY{n}{taille} \PY{o+ow}{in} \PY{n+nb}{enumerate}\PY{p}{(}\PY{n}{liste\PYZus{}taille}\PY{p}{)}\PY{p}{:}
             \PY{n}{liste} \PY{o}{=} \PY{n+nb}{list}\PY{p}{(}\PY{n+nb}{range}\PY{p}{(}\PY{n}{taille}\PY{p}{)}\PY{p}{)}
             \PY{k}{for} \PY{n}{j}\PY{p}{,} \PY{n}{tri} \PY{o+ow}{in} \PY{n+nb}{enumerate}\PY{p}{(}\PY{n}{liste\PYZus{}tris}\PY{p}{)}\PY{p}{:}
                 \PY{n}{liste\PYZus{}temps}\PY{p}{[}\PY{n}{j}\PY{p}{]}\PY{p}{[}\PY{n}{i}\PY{p}{]} \PY{o}{=} \PY{n}{timetest}\PY{p}{(}\PY{n}{tri}\PY{p}{)}\PY{p}{(}\PY{n}{liste}\PY{p}{[}\PY{p}{:}\PY{p}{]}\PY{p}{)}
         \PY{k}{for} \PY{n}{k}\PY{p}{,} \PY{n}{temps} \PY{o+ow}{in} \PY{n+nb}{enumerate}\PY{p}{(}\PY{n}{liste\PYZus{}temps}\PY{p}{)}\PY{p}{:}
             \PY{n}{plt}\PY{o}{.}\PY{n}{loglog}\PY{p}{(}\PY{n}{liste\PYZus{}taille}\PY{p}{,} \PY{n}{temps}\PY{p}{,} \PY{n}{label}\PY{o}{=}\PY{n}{liste\PYZus{}tris}\PY{p}{[}\PY{n}{k}\PY{p}{]}\PY{o}{.}\PY{n+nv+vm}{\PYZus{}\PYZus{}name\PYZus{}\PYZus{}}\PY{p}{,} \PY{n}{marker}\PY{o}{=}\PY{l+s+s1}{\PYZsq{}}\PY{l+s+s1}{o}\PY{l+s+s1}{\PYZsq{}}\PY{p}{)}
         \PY{n}{plt}\PY{o}{.}\PY{n}{legend}\PY{p}{(}\PY{p}{)}
         \PY{n}{plt}\PY{o}{.}\PY{n}{savefig}\PY{p}{(}\PY{l+s+s1}{\PYZsq{}}\PY{l+s+s1}{test\PYZhy{}tris\PYZhy{}listes\PYZhy{}deja\PYZhy{}tri.png}\PY{l+s+s1}{\PYZsq{}}\PY{p}{)}
\end{Verbatim}


    \begin{center}
    \adjustimage{max size={0.9\linewidth}{0.9\paperheight}}{output_53_0.png}
    \end{center}
    { \hspace*{\fill} \\}
    

    % Add a bibliography block to the postdoc
    
    
    
    \end{document}
