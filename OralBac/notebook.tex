
% Default to the notebook output style

    


% Inherit from the specified cell style.




    
\documentclass[11pt]{article}

    
    
    \usepackage[T1]{fontenc}
    % Nicer default font (+ math font) than Computer Modern for most use cases
    \usepackage{mathpazo}

    % Basic figure setup, for now with no caption control since it's done
    % automatically by Pandoc (which extracts ![](path) syntax from Markdown).
    \usepackage{graphicx}
    % We will generate all images so they have a width \maxwidth. This means
    % that they will get their normal width if they fit onto the page, but
    % are scaled down if they would overflow the margins.
    \makeatletter
    \def\maxwidth{\ifdim\Gin@nat@width>\linewidth\linewidth
    \else\Gin@nat@width\fi}
    \makeatother
    \let\Oldincludegraphics\includegraphics
    % Set max figure width to be 80% of text width, for now hardcoded.
    \renewcommand{\includegraphics}[1]{\Oldincludegraphics[width=.8\maxwidth]{#1}}
    % Ensure that by default, figures have no caption (until we provide a
    % proper Figure object with a Caption API and a way to capture that
    % in the conversion process - todo).
    \usepackage{caption}
    \DeclareCaptionLabelFormat{nolabel}{}
    \captionsetup{labelformat=nolabel}

    \usepackage{adjustbox} % Used to constrain images to a maximum size 
    \usepackage{xcolor} % Allow colors to be defined
    \usepackage{enumerate} % Needed for markdown enumerations to work
    \usepackage{geometry} % Used to adjust the document margins
    \usepackage{amsmath} % Equations
    \usepackage{amssymb} % Equations
    \usepackage{textcomp} % defines textquotesingle
    % Hack from http://tex.stackexchange.com/a/47451/13684:
    \AtBeginDocument{%
        \def\PYZsq{\textquotesingle}% Upright quotes in Pygmentized code
    }
    \usepackage{upquote} % Upright quotes for verbatim code
    \usepackage{eurosym} % defines \euro
    \usepackage[mathletters]{ucs} % Extended unicode (utf-8) support
    \usepackage[utf8x]{inputenc} % Allow utf-8 characters in the tex document
    \usepackage{fancyvrb} % verbatim replacement that allows latex
    \usepackage{grffile} % extends the file name processing of package graphics 
                         % to support a larger range 
    % The hyperref package gives us a pdf with properly built
    % internal navigation ('pdf bookmarks' for the table of contents,
    % internal cross-reference links, web links for URLs, etc.)
    \usepackage{hyperref}
    \usepackage{longtable} % longtable support required by pandoc >1.10
    \usepackage{booktabs}  % table support for pandoc > 1.12.2
    \usepackage[inline]{enumitem} % IRkernel/repr support (it uses the enumerate* environment)
    \usepackage[normalem]{ulem} % ulem is needed to support strikethroughs (\sout)
                                % normalem makes italics be italics, not underlines
    

    
    
    % Colors for the hyperref package
    \definecolor{urlcolor}{rgb}{0,.145,.698}
    \definecolor{linkcolor}{rgb}{.71,0.21,0.01}
    \definecolor{citecolor}{rgb}{.12,.54,.11}

    % ANSI colors
    \definecolor{ansi-black}{HTML}{3E424D}
    \definecolor{ansi-black-intense}{HTML}{282C36}
    \definecolor{ansi-red}{HTML}{E75C58}
    \definecolor{ansi-red-intense}{HTML}{B22B31}
    \definecolor{ansi-green}{HTML}{00A250}
    \definecolor{ansi-green-intense}{HTML}{007427}
    \definecolor{ansi-yellow}{HTML}{DDB62B}
    \definecolor{ansi-yellow-intense}{HTML}{B27D12}
    \definecolor{ansi-blue}{HTML}{208FFB}
    \definecolor{ansi-blue-intense}{HTML}{0065CA}
    \definecolor{ansi-magenta}{HTML}{D160C4}
    \definecolor{ansi-magenta-intense}{HTML}{A03196}
    \definecolor{ansi-cyan}{HTML}{60C6C8}
    \definecolor{ansi-cyan-intense}{HTML}{258F8F}
    \definecolor{ansi-white}{HTML}{C5C1B4}
    \definecolor{ansi-white-intense}{HTML}{A1A6B2}

    % commands and environments needed by pandoc snippets
    % extracted from the output of `pandoc -s`
    \providecommand{\tightlist}{%
      \setlength{\itemsep}{0pt}\setlength{\parskip}{0pt}}
    \DefineVerbatimEnvironment{Highlighting}{Verbatim}{commandchars=\\\{\}}
    % Add ',fontsize=\small' for more characters per line
    \newenvironment{Shaded}{}{}
    \newcommand{\KeywordTok}[1]{\textcolor[rgb]{0.00,0.44,0.13}{\textbf{{#1}}}}
    \newcommand{\DataTypeTok}[1]{\textcolor[rgb]{0.56,0.13,0.00}{{#1}}}
    \newcommand{\DecValTok}[1]{\textcolor[rgb]{0.25,0.63,0.44}{{#1}}}
    \newcommand{\BaseNTok}[1]{\textcolor[rgb]{0.25,0.63,0.44}{{#1}}}
    \newcommand{\FloatTok}[1]{\textcolor[rgb]{0.25,0.63,0.44}{{#1}}}
    \newcommand{\CharTok}[1]{\textcolor[rgb]{0.25,0.44,0.63}{{#1}}}
    \newcommand{\StringTok}[1]{\textcolor[rgb]{0.25,0.44,0.63}{{#1}}}
    \newcommand{\CommentTok}[1]{\textcolor[rgb]{0.38,0.63,0.69}{\textit{{#1}}}}
    \newcommand{\OtherTok}[1]{\textcolor[rgb]{0.00,0.44,0.13}{{#1}}}
    \newcommand{\AlertTok}[1]{\textcolor[rgb]{1.00,0.00,0.00}{\textbf{{#1}}}}
    \newcommand{\FunctionTok}[1]{\textcolor[rgb]{0.02,0.16,0.49}{{#1}}}
    \newcommand{\RegionMarkerTok}[1]{{#1}}
    \newcommand{\ErrorTok}[1]{\textcolor[rgb]{1.00,0.00,0.00}{\textbf{{#1}}}}
    \newcommand{\NormalTok}[1]{{#1}}
    
    % Additional commands for more recent versions of Pandoc
    \newcommand{\ConstantTok}[1]{\textcolor[rgb]{0.53,0.00,0.00}{{#1}}}
    \newcommand{\SpecialCharTok}[1]{\textcolor[rgb]{0.25,0.44,0.63}{{#1}}}
    \newcommand{\VerbatimStringTok}[1]{\textcolor[rgb]{0.25,0.44,0.63}{{#1}}}
    \newcommand{\SpecialStringTok}[1]{\textcolor[rgb]{0.73,0.40,0.53}{{#1}}}
    \newcommand{\ImportTok}[1]{{#1}}
    \newcommand{\DocumentationTok}[1]{\textcolor[rgb]{0.73,0.13,0.13}{\textit{{#1}}}}
    \newcommand{\AnnotationTok}[1]{\textcolor[rgb]{0.38,0.63,0.69}{\textbf{\textit{{#1}}}}}
    \newcommand{\CommentVarTok}[1]{\textcolor[rgb]{0.38,0.63,0.69}{\textbf{\textit{{#1}}}}}
    \newcommand{\VariableTok}[1]{\textcolor[rgb]{0.10,0.09,0.49}{{#1}}}
    \newcommand{\ControlFlowTok}[1]{\textcolor[rgb]{0.00,0.44,0.13}{\textbf{{#1}}}}
    \newcommand{\OperatorTok}[1]{\textcolor[rgb]{0.40,0.40,0.40}{{#1}}}
    \newcommand{\BuiltInTok}[1]{{#1}}
    \newcommand{\ExtensionTok}[1]{{#1}}
    \newcommand{\PreprocessorTok}[1]{\textcolor[rgb]{0.74,0.48,0.00}{{#1}}}
    \newcommand{\AttributeTok}[1]{\textcolor[rgb]{0.49,0.56,0.16}{{#1}}}
    \newcommand{\InformationTok}[1]{\textcolor[rgb]{0.38,0.63,0.69}{\textbf{\textit{{#1}}}}}
    \newcommand{\WarningTok}[1]{\textcolor[rgb]{0.38,0.63,0.69}{\textbf{\textit{{#1}}}}}
    
    
    % Define a nice break command that doesn't care if a line doesn't already
    % exist.
    \def\br{\hspace*{\fill} \\* }
    % Math Jax compatability definitions
    \def\gt{>}
    \def\lt{<}
    % Document parameters
    \title{Corrige-Oral-Bac}
    
    
    

    % Pygments definitions
    
\makeatletter
\def\PY@reset{\let\PY@it=\relax \let\PY@bf=\relax%
    \let\PY@ul=\relax \let\PY@tc=\relax%
    \let\PY@bc=\relax \let\PY@ff=\relax}
\def\PY@tok#1{\csname PY@tok@#1\endcsname}
\def\PY@toks#1+{\ifx\relax#1\empty\else%
    \PY@tok{#1}\expandafter\PY@toks\fi}
\def\PY@do#1{\PY@bc{\PY@tc{\PY@ul{%
    \PY@it{\PY@bf{\PY@ff{#1}}}}}}}
\def\PY#1#2{\PY@reset\PY@toks#1+\relax+\PY@do{#2}}

\expandafter\def\csname PY@tok@w\endcsname{\def\PY@tc##1{\textcolor[rgb]{0.73,0.73,0.73}{##1}}}
\expandafter\def\csname PY@tok@c\endcsname{\let\PY@it=\textit\def\PY@tc##1{\textcolor[rgb]{0.25,0.50,0.50}{##1}}}
\expandafter\def\csname PY@tok@cp\endcsname{\def\PY@tc##1{\textcolor[rgb]{0.74,0.48,0.00}{##1}}}
\expandafter\def\csname PY@tok@k\endcsname{\let\PY@bf=\textbf\def\PY@tc##1{\textcolor[rgb]{0.00,0.50,0.00}{##1}}}
\expandafter\def\csname PY@tok@kp\endcsname{\def\PY@tc##1{\textcolor[rgb]{0.00,0.50,0.00}{##1}}}
\expandafter\def\csname PY@tok@kt\endcsname{\def\PY@tc##1{\textcolor[rgb]{0.69,0.00,0.25}{##1}}}
\expandafter\def\csname PY@tok@o\endcsname{\def\PY@tc##1{\textcolor[rgb]{0.40,0.40,0.40}{##1}}}
\expandafter\def\csname PY@tok@ow\endcsname{\let\PY@bf=\textbf\def\PY@tc##1{\textcolor[rgb]{0.67,0.13,1.00}{##1}}}
\expandafter\def\csname PY@tok@nb\endcsname{\def\PY@tc##1{\textcolor[rgb]{0.00,0.50,0.00}{##1}}}
\expandafter\def\csname PY@tok@nf\endcsname{\def\PY@tc##1{\textcolor[rgb]{0.00,0.00,1.00}{##1}}}
\expandafter\def\csname PY@tok@nc\endcsname{\let\PY@bf=\textbf\def\PY@tc##1{\textcolor[rgb]{0.00,0.00,1.00}{##1}}}
\expandafter\def\csname PY@tok@nn\endcsname{\let\PY@bf=\textbf\def\PY@tc##1{\textcolor[rgb]{0.00,0.00,1.00}{##1}}}
\expandafter\def\csname PY@tok@ne\endcsname{\let\PY@bf=\textbf\def\PY@tc##1{\textcolor[rgb]{0.82,0.25,0.23}{##1}}}
\expandafter\def\csname PY@tok@nv\endcsname{\def\PY@tc##1{\textcolor[rgb]{0.10,0.09,0.49}{##1}}}
\expandafter\def\csname PY@tok@no\endcsname{\def\PY@tc##1{\textcolor[rgb]{0.53,0.00,0.00}{##1}}}
\expandafter\def\csname PY@tok@nl\endcsname{\def\PY@tc##1{\textcolor[rgb]{0.63,0.63,0.00}{##1}}}
\expandafter\def\csname PY@tok@ni\endcsname{\let\PY@bf=\textbf\def\PY@tc##1{\textcolor[rgb]{0.60,0.60,0.60}{##1}}}
\expandafter\def\csname PY@tok@na\endcsname{\def\PY@tc##1{\textcolor[rgb]{0.49,0.56,0.16}{##1}}}
\expandafter\def\csname PY@tok@nt\endcsname{\let\PY@bf=\textbf\def\PY@tc##1{\textcolor[rgb]{0.00,0.50,0.00}{##1}}}
\expandafter\def\csname PY@tok@nd\endcsname{\def\PY@tc##1{\textcolor[rgb]{0.67,0.13,1.00}{##1}}}
\expandafter\def\csname PY@tok@s\endcsname{\def\PY@tc##1{\textcolor[rgb]{0.73,0.13,0.13}{##1}}}
\expandafter\def\csname PY@tok@sd\endcsname{\let\PY@it=\textit\def\PY@tc##1{\textcolor[rgb]{0.73,0.13,0.13}{##1}}}
\expandafter\def\csname PY@tok@si\endcsname{\let\PY@bf=\textbf\def\PY@tc##1{\textcolor[rgb]{0.73,0.40,0.53}{##1}}}
\expandafter\def\csname PY@tok@se\endcsname{\let\PY@bf=\textbf\def\PY@tc##1{\textcolor[rgb]{0.73,0.40,0.13}{##1}}}
\expandafter\def\csname PY@tok@sr\endcsname{\def\PY@tc##1{\textcolor[rgb]{0.73,0.40,0.53}{##1}}}
\expandafter\def\csname PY@tok@ss\endcsname{\def\PY@tc##1{\textcolor[rgb]{0.10,0.09,0.49}{##1}}}
\expandafter\def\csname PY@tok@sx\endcsname{\def\PY@tc##1{\textcolor[rgb]{0.00,0.50,0.00}{##1}}}
\expandafter\def\csname PY@tok@m\endcsname{\def\PY@tc##1{\textcolor[rgb]{0.40,0.40,0.40}{##1}}}
\expandafter\def\csname PY@tok@gh\endcsname{\let\PY@bf=\textbf\def\PY@tc##1{\textcolor[rgb]{0.00,0.00,0.50}{##1}}}
\expandafter\def\csname PY@tok@gu\endcsname{\let\PY@bf=\textbf\def\PY@tc##1{\textcolor[rgb]{0.50,0.00,0.50}{##1}}}
\expandafter\def\csname PY@tok@gd\endcsname{\def\PY@tc##1{\textcolor[rgb]{0.63,0.00,0.00}{##1}}}
\expandafter\def\csname PY@tok@gi\endcsname{\def\PY@tc##1{\textcolor[rgb]{0.00,0.63,0.00}{##1}}}
\expandafter\def\csname PY@tok@gr\endcsname{\def\PY@tc##1{\textcolor[rgb]{1.00,0.00,0.00}{##1}}}
\expandafter\def\csname PY@tok@ge\endcsname{\let\PY@it=\textit}
\expandafter\def\csname PY@tok@gs\endcsname{\let\PY@bf=\textbf}
\expandafter\def\csname PY@tok@gp\endcsname{\let\PY@bf=\textbf\def\PY@tc##1{\textcolor[rgb]{0.00,0.00,0.50}{##1}}}
\expandafter\def\csname PY@tok@go\endcsname{\def\PY@tc##1{\textcolor[rgb]{0.53,0.53,0.53}{##1}}}
\expandafter\def\csname PY@tok@gt\endcsname{\def\PY@tc##1{\textcolor[rgb]{0.00,0.27,0.87}{##1}}}
\expandafter\def\csname PY@tok@err\endcsname{\def\PY@bc##1{\setlength{\fboxsep}{0pt}\fcolorbox[rgb]{1.00,0.00,0.00}{1,1,1}{\strut ##1}}}
\expandafter\def\csname PY@tok@kc\endcsname{\let\PY@bf=\textbf\def\PY@tc##1{\textcolor[rgb]{0.00,0.50,0.00}{##1}}}
\expandafter\def\csname PY@tok@kd\endcsname{\let\PY@bf=\textbf\def\PY@tc##1{\textcolor[rgb]{0.00,0.50,0.00}{##1}}}
\expandafter\def\csname PY@tok@kn\endcsname{\let\PY@bf=\textbf\def\PY@tc##1{\textcolor[rgb]{0.00,0.50,0.00}{##1}}}
\expandafter\def\csname PY@tok@kr\endcsname{\let\PY@bf=\textbf\def\PY@tc##1{\textcolor[rgb]{0.00,0.50,0.00}{##1}}}
\expandafter\def\csname PY@tok@bp\endcsname{\def\PY@tc##1{\textcolor[rgb]{0.00,0.50,0.00}{##1}}}
\expandafter\def\csname PY@tok@fm\endcsname{\def\PY@tc##1{\textcolor[rgb]{0.00,0.00,1.00}{##1}}}
\expandafter\def\csname PY@tok@vc\endcsname{\def\PY@tc##1{\textcolor[rgb]{0.10,0.09,0.49}{##1}}}
\expandafter\def\csname PY@tok@vg\endcsname{\def\PY@tc##1{\textcolor[rgb]{0.10,0.09,0.49}{##1}}}
\expandafter\def\csname PY@tok@vi\endcsname{\def\PY@tc##1{\textcolor[rgb]{0.10,0.09,0.49}{##1}}}
\expandafter\def\csname PY@tok@vm\endcsname{\def\PY@tc##1{\textcolor[rgb]{0.10,0.09,0.49}{##1}}}
\expandafter\def\csname PY@tok@sa\endcsname{\def\PY@tc##1{\textcolor[rgb]{0.73,0.13,0.13}{##1}}}
\expandafter\def\csname PY@tok@sb\endcsname{\def\PY@tc##1{\textcolor[rgb]{0.73,0.13,0.13}{##1}}}
\expandafter\def\csname PY@tok@sc\endcsname{\def\PY@tc##1{\textcolor[rgb]{0.73,0.13,0.13}{##1}}}
\expandafter\def\csname PY@tok@dl\endcsname{\def\PY@tc##1{\textcolor[rgb]{0.73,0.13,0.13}{##1}}}
\expandafter\def\csname PY@tok@s2\endcsname{\def\PY@tc##1{\textcolor[rgb]{0.73,0.13,0.13}{##1}}}
\expandafter\def\csname PY@tok@sh\endcsname{\def\PY@tc##1{\textcolor[rgb]{0.73,0.13,0.13}{##1}}}
\expandafter\def\csname PY@tok@s1\endcsname{\def\PY@tc##1{\textcolor[rgb]{0.73,0.13,0.13}{##1}}}
\expandafter\def\csname PY@tok@mb\endcsname{\def\PY@tc##1{\textcolor[rgb]{0.40,0.40,0.40}{##1}}}
\expandafter\def\csname PY@tok@mf\endcsname{\def\PY@tc##1{\textcolor[rgb]{0.40,0.40,0.40}{##1}}}
\expandafter\def\csname PY@tok@mh\endcsname{\def\PY@tc##1{\textcolor[rgb]{0.40,0.40,0.40}{##1}}}
\expandafter\def\csname PY@tok@mi\endcsname{\def\PY@tc##1{\textcolor[rgb]{0.40,0.40,0.40}{##1}}}
\expandafter\def\csname PY@tok@il\endcsname{\def\PY@tc##1{\textcolor[rgb]{0.40,0.40,0.40}{##1}}}
\expandafter\def\csname PY@tok@mo\endcsname{\def\PY@tc##1{\textcolor[rgb]{0.40,0.40,0.40}{##1}}}
\expandafter\def\csname PY@tok@ch\endcsname{\let\PY@it=\textit\def\PY@tc##1{\textcolor[rgb]{0.25,0.50,0.50}{##1}}}
\expandafter\def\csname PY@tok@cm\endcsname{\let\PY@it=\textit\def\PY@tc##1{\textcolor[rgb]{0.25,0.50,0.50}{##1}}}
\expandafter\def\csname PY@tok@cpf\endcsname{\let\PY@it=\textit\def\PY@tc##1{\textcolor[rgb]{0.25,0.50,0.50}{##1}}}
\expandafter\def\csname PY@tok@c1\endcsname{\let\PY@it=\textit\def\PY@tc##1{\textcolor[rgb]{0.25,0.50,0.50}{##1}}}
\expandafter\def\csname PY@tok@cs\endcsname{\let\PY@it=\textit\def\PY@tc##1{\textcolor[rgb]{0.25,0.50,0.50}{##1}}}

\def\PYZbs{\char`\\}
\def\PYZus{\char`\_}
\def\PYZob{\char`\{}
\def\PYZcb{\char`\}}
\def\PYZca{\char`\^}
\def\PYZam{\char`\&}
\def\PYZlt{\char`\<}
\def\PYZgt{\char`\>}
\def\PYZsh{\char`\#}
\def\PYZpc{\char`\%}
\def\PYZdl{\char`\$}
\def\PYZhy{\char`\-}
\def\PYZsq{\char`\'}
\def\PYZdq{\char`\"}
\def\PYZti{\char`\~}
% for compatibility with earlier versions
\def\PYZat{@}
\def\PYZlb{[}
\def\PYZrb{]}
\makeatother


    % Exact colors from NB
    \definecolor{incolor}{rgb}{0.0, 0.0, 0.5}
    \definecolor{outcolor}{rgb}{0.545, 0.0, 0.0}



    
    % Prevent overflowing lines due to hard-to-break entities
    \sloppy 
    % Setup hyperref package
    \hypersetup{
      breaklinks=true,  % so long urls are correctly broken across lines
      colorlinks=true,
      urlcolor=urlcolor,
      linkcolor=linkcolor,
      citecolor=citecolor,
      }
    % Slightly bigger margins than the latex defaults
    
    \geometry{verbose,tmargin=1in,bmargin=1in,lmargin=1in,rmargin=1in}
    
    

    \begin{document}
    
    
    \maketitle
    
    

    
    \hypertarget{corriguxe9s-des-exercices-de-pruxe9paration-uxe0-loral-du-bac}{%
\section{Corrigés des exercices de préparation à l'oral du
Bac}\label{corriguxe9s-des-exercices-de-pruxe9paration-uxe0-loral-du-bac}}

Les énoncés des exercices se trouvent sur :

-\href{OralBac/Preparation-oral-bac-.pdf}{version pdf}

-\href{OralBac/Preparation-oral-bac-git.md}{version Github markdown}

-\href{OralBac/Preparation-oral-bac-slidy.html}{version diaporama HTML}

-\href{OralBac/Preparation-oral-bac-.html}{version simple HTML}

-\href{https://mybinder.org/v2/gh/frederic-junier/ISN/master?filepath=OralBac/Corrige-Oral-Bac.ipynb}{corrigé
mis à jour régulièrement}

    \hypertarget{exercice-6}{%
\subsection{Exercice 6}\label{exercice-6}}

Pour le diplôme du baccalauréat, si on note \(m\) la moyenne du
candidat, quatre mentions sont possibles : \emph{Passable} si
\(10 \leqslant m < 12\), \emph{Assez bien} si \(12 \leqslant m < 14\),
\emph{Bien} si \(14 \leqslant m < 16\) et \emph{Très bien} sinon.
Recopier et compléter le script Python ci-dessous pour qu'il affiche la
mention d'un candidat admis (on suppose sa moyenne supérieure ou égale à
10).

\begin{Shaded}
\begin{Highlighting}[]
\NormalTok{m }\OperatorTok{=} \BuiltInTok{float}\NormalTok{(}\BuiltInTok{input}\NormalTok{(}\StringTok{\textquotesingle{}Moyenne du candidat ? \textquotesingle{}}\NormalTok{))}
\ControlFlowTok{if} \DecValTok{10} \OperatorTok{<=}\NormalTok{ m }\OperatorTok{<} \DecValTok{12}\NormalTok{:}
    \BuiltInTok{print}\NormalTok{(}\StringTok{"Passable"}\NormalTok{)}
\CommentTok{\#to be completed}
\end{Highlighting}
\end{Shaded}

    \textbf{Réponse ci-dessous :}

    \begin{Verbatim}[commandchars=\\\{\}]
{\color{incolor}In [{\color{incolor}3}]:} \PY{n}{m} \PY{o}{=} \PY{n+nb}{float}\PY{p}{(}\PY{n+nb}{input}\PY{p}{(}\PY{l+s+s1}{\PYZsq{}}\PY{l+s+s1}{Moyenne du candidat ? }\PY{l+s+s1}{\PYZsq{}}\PY{p}{)}\PY{p}{)}
        \PY{k}{if} \PY{l+m+mi}{10} \PY{o}{\PYZlt{}}\PY{o}{=} \PY{n}{m} \PY{o}{\PYZlt{}} \PY{l+m+mi}{12}\PY{p}{:}
            \PY{n+nb}{print}\PY{p}{(}\PY{l+s+s2}{\PYZdq{}}\PY{l+s+s2}{Passable}\PY{l+s+s2}{\PYZdq{}}\PY{p}{)}
        \PY{k}{elif} \PY{l+m+mi}{12} \PY{o}{\PYZlt{}}\PY{o}{=} \PY{n}{m} \PY{o}{\PYZlt{}} \PY{l+m+mi}{14}\PY{p}{:}
            \PY{n+nb}{print}\PY{p}{(}\PY{l+s+s2}{\PYZdq{}}\PY{l+s+s2}{Assez Bien}\PY{l+s+s2}{\PYZdq{}}\PY{p}{)}
        \PY{k}{elif} \PY{l+m+mi}{14} \PY{o}{\PYZlt{}}\PY{o}{=} \PY{n}{m} \PY{o}{\PYZlt{}} \PY{l+m+mi}{16}\PY{p}{:}
            \PY{n+nb}{print}\PY{p}{(}\PY{l+s+s2}{\PYZdq{}}\PY{l+s+s2}{Bien}\PY{l+s+s2}{\PYZdq{}}\PY{p}{)}
        \PY{k}{else}\PY{p}{:}
            \PY{n+nb}{print}\PY{p}{(}\PY{l+s+s2}{\PYZdq{}}\PY{l+s+s2}{Très Bien}\PY{l+s+s2}{\PYZdq{}}\PY{p}{)}
\end{Verbatim}


    \begin{Verbatim}[commandchars=\\\{\}]
Moyenne du candidat ? 15
Bien

    \end{Verbatim}

    \hypertarget{exercice-8}{%
\subsection{Exercice 8}\label{exercice-8}}

\emph{Auteur : Christophe Beasse}

Que contient la variable \texttt{a} si on exécute ce script ?

\begin{Shaded}
\begin{Highlighting}[]
\KeywordTok{def}\NormalTok{ carre(val):}
  \ControlFlowTok{return}\NormalTok{ val}\OperatorTok{*}\NormalTok{val}

\KeywordTok{def}\NormalTok{ inc(val):}
  \ControlFlowTok{return}\NormalTok{ val }\OperatorTok{+} \DecValTok{1}

\NormalTok{a }\OperatorTok{=}\NormalTok{ carre(inc(}\FloatTok{3.0}\NormalTok{))}
\end{Highlighting}
\end{Shaded}

\begin{enumerate}
\def\labelenumi{\arabic{enumi}.}
\tightlist
\item
  9.0
\item
  12.0
\item
  10.0
\item
  16.0
\end{enumerate}

    \textbf{Réponse :} 4. 16

    \hypertarget{exercice-7}{%
\subsection{Exercice 7}\label{exercice-7}}

\emph{Auteur : Nicolas Revéret}

On considère le code suivant :

\begin{Shaded}
\begin{Highlighting}[]
\KeywordTok{def}\NormalTok{ f(tab):}
  \ControlFlowTok{for}\NormalTok{ i }\KeywordTok{in} \BuiltInTok{range}\NormalTok{(}\BuiltInTok{len}\NormalTok{(tab)}\OperatorTok{//}\DecValTok{2}\NormalTok{):}
\NormalTok{    tab[i],tab[}\OperatorTok{{-}}\NormalTok{i}\DecValTok{{-}1}\NormalTok{] }\OperatorTok{=}\NormalTok{ tab[}\OperatorTok{{-}}\NormalTok{i}\DecValTok{{-}1}\NormalTok{],tab[i]}
\end{Highlighting}
\end{Shaded}

Après les lignes suivantes :

\begin{Shaded}
\begin{Highlighting}[]
\NormalTok{tab }\OperatorTok{=}\NormalTok{ [}\DecValTok{2}\NormalTok{,}\DecValTok{3}\NormalTok{,}\DecValTok{4}\NormalTok{,}\DecValTok{5}\NormalTok{,}\DecValTok{7}\NormalTok{,}\DecValTok{8}\NormalTok{]}
\NormalTok{f(tab)}
\end{Highlighting}
\end{Shaded}

Quelle est la valeur de la variable \texttt{tab} ?

Réponses :

\begin{enumerate}
\def\labelenumi{\arabic{enumi}.}
\tightlist
\item
  \texttt{{[}2,3,4,5,7,8{]}}
\item
  \texttt{{[}5,7,8,2,3,4{]}}
\item
  \texttt{{[}8,7,5,4,3,2{]}}
\item
  \texttt{{[}4,3,2,8,7,5{]}}
\end{enumerate}

    \textbf{Réponse} : cette fonction retourne la liste inversée
\texttt{{[}8,7,5,4,3,2{]}} et donc la réponse est la 3.

    \hypertarget{exercice-9}{%
\subsection{Exercice 9}\label{exercice-9}}

\emph{Auteur : Christophe Beasse}

Soit la liste suivante :
\texttt{liste\_pays\ ={[}"France","Allemagne","Italie","Belgique"{]}}

Que renvoie : \texttt{liste\_pays{[}2{]}} ?

Réponses :

\begin{enumerate}
\def\labelenumi{\arabic{enumi}.}
\tightlist
\item
  \texttt{"France"}
\item
  \texttt{"Allemagne"}
\item
  \texttt{"Italie"}
\item
  \texttt{"Belgique"}
\end{enumerate}

    \textbf{Réponse} : 3. ``Italie''

    \hypertarget{exercice-12}{%
\subsection{Exercice 12}\label{exercice-12}}

\emph{Auteur : Nicolas Revéret}

On dispose d'un tableau d'entiers, ordonné en ordre croissant.

On désire connaître le nombre de valeurs distinctes contenues dans ce
tableau.

Quelle est la fonction qui ne convient pas ?

Réponses :

\begin{enumerate}
\def\labelenumi{\arabic{enumi}.}
\tightlist
\item
  Réponse 1
\end{enumerate}

\begin{Shaded}
\begin{Highlighting}[]
\KeywordTok{def}\NormalTok{ compte(t):}
\NormalTok{    cpt }\OperatorTok{=} \DecValTok{1}
    \ControlFlowTok{for}\NormalTok{ i }\KeywordTok{in} \BuiltInTok{range}\NormalTok{(}\DecValTok{1}\NormalTok{,}\BuiltInTok{len}\NormalTok{(t)):}
        \ControlFlowTok{if}\NormalTok{ t[i] }\OperatorTok{!=}\NormalTok{ t[i}\DecValTok{{-}1}\NormalTok{]:}
\NormalTok{            cpt }\OperatorTok{=}\NormalTok{ cpt }\OperatorTok{+} \DecValTok{1}
    \ControlFlowTok{return}\NormalTok{ cpt}
\end{Highlighting}
\end{Shaded}

\begin{enumerate}
\def\labelenumi{\arabic{enumi}.}
\setcounter{enumi}{1}
\tightlist
\item
  Réponse 2
\end{enumerate}

\begin{Shaded}
\begin{Highlighting}[]
    \KeywordTok{def}\NormalTok{ compte(t):}
\NormalTok{      cpt }\OperatorTok{=} \DecValTok{0}
      \ControlFlowTok{for}\NormalTok{ i }\KeywordTok{in} \BuiltInTok{range}\NormalTok{(}\DecValTok{0}\NormalTok{,}\BuiltInTok{len}\NormalTok{(t)}\OperatorTok{{-}}\DecValTok{1}\NormalTok{):}
\NormalTok{        cpt }\OperatorTok{=}\NormalTok{ cpt }\OperatorTok{+} \BuiltInTok{int}\NormalTok{(t[i] }\OperatorTok{!=}\NormalTok{ t[i}\OperatorTok{+}\DecValTok{1}\NormalTok{])}
      \ControlFlowTok{return}\NormalTok{ cpt}
\end{Highlighting}
\end{Shaded}

\begin{enumerate}
\def\labelenumi{\arabic{enumi}.}
\setcounter{enumi}{2}
\tightlist
\item
  Réponse 3
\end{enumerate}

\begin{Shaded}
\begin{Highlighting}[]
    \KeywordTok{def}\NormalTok{ compte(t):}
\NormalTok{      cpt }\OperatorTok{=} \DecValTok{0}
      \ControlFlowTok{for}\NormalTok{ i }\KeywordTok{in} \BuiltInTok{range}\NormalTok{(}\DecValTok{0}\NormalTok{,}\BuiltInTok{len}\NormalTok{(t)}\OperatorTok{{-}}\DecValTok{1}\NormalTok{):}
        \ControlFlowTok{if}\NormalTok{ t[i] }\OperatorTok{!=}\NormalTok{ t[i}\OperatorTok{+}\DecValTok{1}\NormalTok{]:}
\NormalTok{          cpt }\OperatorTok{=}\NormalTok{ cpt }\OperatorTok{+} \DecValTok{1}
      \ControlFlowTok{return}\NormalTok{ cpt}\OperatorTok{+}\DecValTok{1}
\end{Highlighting}
\end{Shaded}

    \textbf{Corrigé: la bonne réponse est la 3, voir ci-dessous}

    \begin{Verbatim}[commandchars=\\\{\}]
{\color{incolor}In [{\color{incolor}10}]:} \PY{k}{def} \PY{n+nf}{compte}\PY{p}{(}\PY{n}{t}\PY{p}{)}\PY{p}{:}
             \PY{n}{cpt} \PY{o}{=} \PY{l+m+mi}{1}
             \PY{k}{for} \PY{n}{i} \PY{o+ow}{in} \PY{n+nb}{range}\PY{p}{(}\PY{l+m+mi}{1}\PY{p}{,}\PY{n+nb}{len}\PY{p}{(}\PY{n}{t}\PY{p}{)}\PY{p}{)}\PY{p}{:}       
                 \PY{k}{if} \PY{n}{t}\PY{p}{[}\PY{n}{i}\PY{p}{]} \PY{o}{!=} \PY{n}{t}\PY{p}{[}\PY{n}{i}\PY{o}{\PYZhy{}}\PY{l+m+mi}{1}\PY{p}{]}\PY{p}{:}
                     \PY{n}{cpt} \PY{o}{=} \PY{n}{cpt} \PY{o}{+} \PY{l+m+mi}{1}
                 \PY{n+nb}{print}\PY{p}{(}\PY{n}{t}\PY{p}{[}\PY{n}{i}\PY{p}{]}\PY{p}{,} \PY{n}{t}\PY{p}{[}\PY{n}{i}\PY{o}{\PYZhy{}}\PY{l+m+mi}{1}\PY{p}{]}\PY{p}{,} \PY{n}{cpt}\PY{p}{)}
             \PY{k}{return} \PY{n}{cpt}
         
         
         \PY{k}{def} \PY{n+nf}{compte2}\PY{p}{(}\PY{n}{t}\PY{p}{)}\PY{p}{:}
             \PY{n}{cpt} \PY{o}{=} \PY{l+m+mi}{0}
             \PY{k}{for} \PY{n}{i} \PY{o+ow}{in} \PY{n+nb}{range}\PY{p}{(}\PY{l+m+mi}{0}\PY{p}{,}\PY{n+nb}{len}\PY{p}{(}\PY{n}{t}\PY{p}{)}\PY{o}{\PYZhy{}}\PY{l+m+mi}{1}\PY{p}{)}\PY{p}{:}
              \PY{c+c1}{\PYZsh{}cpt = cpt + int(t[i] != t[i+1])}
              \PY{k}{if} \PY{n}{t}\PY{p}{[}\PY{n}{i}\PY{p}{]} \PY{o}{!=} \PY{n}{t}\PY{p}{[}\PY{n}{i}\PY{o}{+}\PY{l+m+mi}{1}\PY{p}{]}\PY{p}{:}
                 \PY{n}{cpt} \PY{o}{=} \PY{n}{cpt} \PY{o}{+} \PY{l+m+mi}{1}
             \PY{k}{return} \PY{n}{cpt}
         
         \PY{k}{def} \PY{n+nf}{compte3}\PY{p}{(}\PY{n}{t}\PY{p}{)}\PY{p}{:}
               \PY{n}{cpt} \PY{o}{=} \PY{l+m+mi}{0}
               \PY{k}{for} \PY{n}{i} \PY{o+ow}{in} \PY{n+nb}{range}\PY{p}{(}\PY{l+m+mi}{0}\PY{p}{,}\PY{n+nb}{len}\PY{p}{(}\PY{n}{t}\PY{p}{)}\PY{o}{\PYZhy{}}\PY{l+m+mi}{1}\PY{p}{)}\PY{p}{:}
                 \PY{k}{if} \PY{n}{t}\PY{p}{[}\PY{n}{i}\PY{p}{]} \PY{o}{!=} \PY{n}{t}\PY{p}{[}\PY{n}{i}\PY{o}{+}\PY{l+m+mi}{1}\PY{p}{]}\PY{p}{:}
                   \PY{n}{cpt} \PY{o}{=} \PY{n}{cpt} \PY{o}{+} \PY{l+m+mi}{1}
               \PY{k}{return} \PY{n}{cpt} \PY{o}{+} \PY{l+m+mi}{1}
\end{Verbatim}


    \begin{Verbatim}[commandchars=\\\{\}]
{\color{incolor}In [{\color{incolor}5}]:} \PY{n}{compte}\PY{p}{(}\PY{p}{[}\PY{l+m+mi}{1}\PY{p}{,}\PY{l+m+mi}{1}\PY{p}{,}\PY{l+m+mi}{2}\PY{p}{,}\PY{l+m+mi}{3}\PY{p}{,}\PY{l+m+mi}{3}\PY{p}{,}\PY{l+m+mi}{4}\PY{p}{]}\PY{p}{)}
\end{Verbatim}


    \begin{Verbatim}[commandchars=\\\{\}]
1 1 1
2 1 2
3 2 3
3 3 3
4 3 4

    \end{Verbatim}

\begin{Verbatim}[commandchars=\\\{\}]
{\color{outcolor}Out[{\color{outcolor}5}]:} 4
\end{Verbatim}
            
    \begin{Verbatim}[commandchars=\\\{\}]
{\color{incolor}In [{\color{incolor}6}]:} \PY{l+m+mi}{0} \PY{o}{!=} \PY{l+m+mi}{2}
\end{Verbatim}


\begin{Verbatim}[commandchars=\\\{\}]
{\color{outcolor}Out[{\color{outcolor}6}]:} True
\end{Verbatim}
            
    \begin{Verbatim}[commandchars=\\\{\}]
{\color{incolor}In [{\color{incolor}7}]:} \PY{l+m+mi}{0} \PY{o}{==} \PY{l+m+mi}{2}
\end{Verbatim}


\begin{Verbatim}[commandchars=\\\{\}]
{\color{outcolor}Out[{\color{outcolor}7}]:} False
\end{Verbatim}
            
    \begin{Verbatim}[commandchars=\\\{\}]
{\color{incolor}In [{\color{incolor}8}]:} \PY{n+nb}{int}\PY{p}{(}\PY{l+m+mi}{0} \PY{o}{!=} \PY{l+m+mi}{2}\PY{p}{)}
\end{Verbatim}


\begin{Verbatim}[commandchars=\\\{\}]
{\color{outcolor}Out[{\color{outcolor}8}]:} 1
\end{Verbatim}
            
    \begin{Verbatim}[commandchars=\\\{\}]
{\color{incolor}In [{\color{incolor}9}]:} \PY{n+nb}{int}\PY{p}{(} \PY{l+m+mi}{0} \PY{o}{==} \PY{l+m+mi}{2}\PY{p}{)}
\end{Verbatim}


\begin{Verbatim}[commandchars=\\\{\}]
{\color{outcolor}Out[{\color{outcolor}9}]:} 0
\end{Verbatim}
            
    \hypertarget{exercice-13}{%
\subsection{Exercice 13}\label{exercice-13}}

\emph{Auteur : Eric Rougier}

Quel est le résultat de l'évaluation de l'expression Python suivante ?

\texttt{{[}2\ **\ n\ for\ n\ in\ range(4){]}}

Réponses :

\begin{enumerate}
\def\labelenumi{\arabic{enumi}.}
\tightlist
\item
  \texttt{{[}0,\ 2,\ 4,\ 6,\ 8{]}}
\item
  \texttt{{[}1,\ 2,\ 4,\ 8{]}}
\item
  \texttt{{[}0,\ 1,\ 4,\ 9{]}}
\item
  \texttt{{[}1,\ 2,\ 4,\ 8,\ 16{]}}
\end{enumerate}

    \textbf{Corrigé} : La réponse est la 2.

    \begin{Verbatim}[commandchars=\\\{\}]
{\color{incolor}In [{\color{incolor}3}]:} \PY{p}{[}\PY{l+m+mi}{2} \PY{o}{*}\PY{o}{*} \PY{n}{n} \PY{k}{for} \PY{n}{n} \PY{o+ow}{in} \PY{n+nb}{range}\PY{p}{(}\PY{l+m+mi}{4}\PY{p}{)}\PY{p}{]}
\end{Verbatim}


\begin{Verbatim}[commandchars=\\\{\}]
{\color{outcolor}Out[{\color{outcolor}3}]:} [1, 2, 4, 8]
\end{Verbatim}
            
    \hypertarget{exercice-14}{%
\subsection{Exercice 14}\label{exercice-14}}

\emph{Auteur : Germain Becker, question n°326 Genumsi}

Quel est le tableau \texttt{t} construit par les instructions suivantes
?

\begin{Shaded}
\begin{Highlighting}[]
\NormalTok{tab }\OperatorTok{=}\NormalTok{ [}\DecValTok{1}\NormalTok{, }\DecValTok{2}\NormalTok{, }\DecValTok{{-}3}\NormalTok{, }\DecValTok{7}\NormalTok{, }\DecValTok{4}\NormalTok{, }\DecValTok{10}\NormalTok{, }\DecValTok{{-}1}\NormalTok{, }\DecValTok{0}\NormalTok{]}
\NormalTok{t }\OperatorTok{=}\NormalTok{ [e }\ControlFlowTok{for}\NormalTok{ e }\KeywordTok{in}\NormalTok{ tab }\ControlFlowTok{if}\NormalTok{ e }\OperatorTok{>=} \DecValTok{0}\NormalTok{]}
\end{Highlighting}
\end{Shaded}

Réponses :

\begin{enumerate}
\def\labelenumi{\arabic{enumi}.}
\tightlist
\item
  \texttt{t\ =\ {[}1,\ 2,\ 7,\ 4,\ 10,\ 0{]}}
\item
  \texttt{t\ =\ {[}e,\ e,\ e,\ e,\ e,\ e{]}}
\item
  \texttt{t\ =\ {[}1,\ 2,\ 7,\ 4,\ 10{]}}
\item
  \texttt{t\ =\ {[}-3,\ -1,\ 0{]}}
\end{enumerate}

    \textbf{Corrigé} : La réponse est la 3.

    \begin{Verbatim}[commandchars=\\\{\}]
{\color{incolor}In [{\color{incolor}5}]:} \PY{n}{tab} \PY{o}{=} \PY{p}{[}\PY{l+m+mi}{1}\PY{p}{,} \PY{l+m+mi}{2}\PY{p}{,} \PY{o}{\PYZhy{}}\PY{l+m+mi}{3}\PY{p}{,} \PY{l+m+mi}{7}\PY{p}{,} \PY{l+m+mi}{4}\PY{p}{,} \PY{l+m+mi}{10}\PY{p}{,} \PY{o}{\PYZhy{}}\PY{l+m+mi}{1}\PY{p}{,} \PY{l+m+mi}{0}\PY{p}{]}
        \PY{n}{t} \PY{o}{=} \PY{p}{[}\PY{n}{e} \PY{k}{for} \PY{n}{e} \PY{o+ow}{in} \PY{n}{tab} \PY{k}{if} \PY{n}{e} \PY{o}{\PYZgt{}}\PY{o}{=} \PY{l+m+mi}{0}\PY{p}{]}
        \PY{n}{t}
\end{Verbatim}


\begin{Verbatim}[commandchars=\\\{\}]
{\color{outcolor}Out[{\color{outcolor}5}]:} [1, 2, 7, 4, 10, 0]
\end{Verbatim}
            
    \hypertarget{exercice-15}{%
\subsection{Exercice 15}\label{exercice-15}}

\emph{Auteur : Germain Becker, question n°339 Genumsi}

On considère le tableau t suivant.

\texttt{t\ =\ {[}{[}1,\ 2,\ 3{]},\ {[}2,\ 3,\ 4{]},\ {[}3,\ 4,\ 5{]},\ {[}4,\ 5,\ 6{]}{]}}

Quelle est la valeur de t{[}1{]}{[}2{]} ?

Réponses :

\begin{enumerate}
\def\labelenumi{\arabic{enumi}.}
\tightlist
\item
  1
\item
  3
\item
  4
\item
  2
\end{enumerate}

    \textbf{Corrigé} : La réponse est la 3.

    \begin{Verbatim}[commandchars=\\\{\}]
{\color{incolor}In [{\color{incolor}8}]:} \PY{n}{t} \PY{o}{=} \PY{p}{[}\PY{p}{[}\PY{l+m+mi}{1}\PY{p}{,} \PY{l+m+mi}{2}\PY{p}{,} \PY{l+m+mi}{3}\PY{p}{]}\PY{p}{,} \PY{p}{[}\PY{l+m+mi}{2}\PY{p}{,} \PY{l+m+mi}{3}\PY{p}{,} \PY{l+m+mi}{4}\PY{p}{]}\PY{p}{,} \PY{p}{[}\PY{l+m+mi}{3}\PY{p}{,} \PY{l+m+mi}{4}\PY{p}{,} \PY{l+m+mi}{5}\PY{p}{]}\PY{p}{,} \PY{p}{[}\PY{l+m+mi}{4}\PY{p}{,} \PY{l+m+mi}{5}\PY{p}{,} \PY{l+m+mi}{6}\PY{p}{]}\PY{p}{]}
        \PY{n}{t}\PY{p}{[}\PY{l+m+mi}{1}\PY{p}{]}\PY{p}{[}\PY{l+m+mi}{2}\PY{p}{]} 
\end{Verbatim}


\begin{Verbatim}[commandchars=\\\{\}]
{\color{outcolor}Out[{\color{outcolor}8}]:} 4
\end{Verbatim}
            
    \hypertarget{exercice-28}{%
\subsection{Exercice 28}\label{exercice-28}}

Le codage en base deux de l'entier 26 en base dix est :

\begin{enumerate}
\def\labelenumi{\arabic{enumi}.}
\tightlist
\item
  11010
\item
  10010
\item
  11001
\item
  110010
\end{enumerate}

    \textbf{Corrigé} : La réponse est la 1.

    \begin{Verbatim}[commandchars=\\\{\}]
{\color{incolor}In [{\color{incolor}10}]:} \PY{n+nb}{bin}\PY{p}{(}\PY{l+m+mi}{26}\PY{p}{)}
\end{Verbatim}


\begin{Verbatim}[commandchars=\\\{\}]
{\color{outcolor}Out[{\color{outcolor}10}]:} '0b11010'
\end{Verbatim}
            
    \hypertarget{exercice-29}{%
\subsection{Exercice 29}\label{exercice-29}}

Le résultat de la somme
\(\overline{101101}^{2} + \overline{101111}^{2}\) est :

\begin{enumerate}
\def\labelenumi{\arabic{enumi}.}
\tightlist
\item
  \(\overline{1100100}^{2}\)
\item
  \(\overline{1110101}^{2}\)
\item
  \(\overline{1011100}^{2}\)
\item
  \(\overline{1111100}^{2}\)
\end{enumerate}

    \textbf{Corrigé} : La réponse est la 3.

    \begin{Verbatim}[commandchars=\\\{\}]
{\color{incolor}In [{\color{incolor}12}]:} \PY{l+m+mb}{0b101101} \PY{o}{+} \PY{l+m+mb}{0b101111}
\end{Verbatim}


\begin{Verbatim}[commandchars=\\\{\}]
{\color{outcolor}Out[{\color{outcolor}12}]:} 92
\end{Verbatim}
            
    \begin{Verbatim}[commandchars=\\\{\}]
{\color{incolor}In [{\color{incolor}13}]:} \PY{n+nb}{bin}\PY{p}{(}\PY{l+m+mi}{92}\PY{p}{)}
\end{Verbatim}


\begin{Verbatim}[commandchars=\\\{\}]
{\color{outcolor}Out[{\color{outcolor}13}]:} '0b1011100'
\end{Verbatim}
            
    \hypertarget{exercice-50}{%
\subsection{Exercice 50}\label{exercice-50}}

Parmi les quatre expressions suivantes, laquelle s'évalue en True ?

Réponses :

\begin{enumerate}
\def\labelenumi{\arabic{enumi}.}
\item
  \textbf{Réponse 1 :} \texttt{False\ and\ (True\ and\ False)}
\item
  \textbf{Réponse 2 :} \texttt{False\ or\ (True\ and\ False)}
\item
  \textbf{Réponse 3 :} \texttt{True\ and\ (True\ and\ False)}
\item
  \textbf{Réponse 4 :} \texttt{True\ or\ (True\ and\ False)}
\end{enumerate}

    \textbf{Corrigé} : La réponse est la 4.

    \begin{Verbatim}[commandchars=\\\{\}]
{\color{incolor}In [{\color{incolor}17}]:} \PY{k}{for} \PY{n}{exp} \PY{o+ow}{in} \PY{p}{[}\PY{l+s+s1}{\PYZsq{}}\PY{l+s+s1}{False and (True and False)}\PY{l+s+s1}{\PYZsq{}}\PY{p}{,} \PY{l+s+s1}{\PYZsq{}}\PY{l+s+s1}{False or (True and False)}\PY{l+s+s1}{\PYZsq{}}\PY{p}{,}\PY{l+s+s1}{\PYZsq{}}\PY{l+s+s1}{True and (True and False)}\PY{l+s+s1}{\PYZsq{}}\PY{p}{,}\PY{l+s+s1}{\PYZsq{}}\PY{l+s+s1}{True or (True and False)}\PY{l+s+s1}{\PYZsq{}}\PY{p}{]}\PY{p}{:}
             \PY{n+nb}{print}\PY{p}{(}\PY{l+s+s2}{\PYZdq{}}\PY{l+s+s2}{Valeur booléenne de }\PY{l+s+s2}{\PYZdq{}}\PY{p}{,} \PY{n}{exp}\PY{p}{,} \PY{l+s+s2}{\PYZdq{}}\PY{l+s+s2}{ = }\PY{l+s+s2}{\PYZdq{}}\PY{p}{,} \PY{n+nb}{eval}\PY{p}{(}\PY{n}{exp}\PY{p}{)}\PY{p}{)}
\end{Verbatim}


    \begin{Verbatim}[commandchars=\\\{\}]
Valeur booléenne de  False and (True and False)  =  False
Valeur booléenne de  False or (True and False)  =  False
Valeur booléenne de  True and (True and False)  =  False
Valeur booléenne de  True or (True and False)  =  True

    \end{Verbatim}

    T est un tableau de nombres entiers non vide. Que représente la valeur
de s~renvoyée par cette fonction ?

\begin{Shaded}
\begin{Highlighting}[]
\KeywordTok{def}\NormalTok{ mystere(T):}
\NormalTok{    s }\OperatorTok{=} \DecValTok{0}
    \ControlFlowTok{for}\NormalTok{ k }\KeywordTok{in}\NormalTok{ T:}
        \ControlFlowTok{if}\NormalTok{ k }\OperatorTok{\%} \DecValTok{2} \OperatorTok{==} \DecValTok{0}\NormalTok{:}
\NormalTok{            s }\OperatorTok{=}\NormalTok{ s}\OperatorTok{+}\NormalTok{k}
    \ControlFlowTok{return}\NormalTok{ s}
\end{Highlighting}
\end{Shaded}

Réponses :

\begin{enumerate}
\def\labelenumi{\arabic{enumi}.}
\item
  \textbf{Réponse 1 :} la somme des valeurs du tableau T
\item
  \textbf{Réponse 2 :} la somme des valeurs positives du tableau T
\item
  \textbf{Réponse 3 :} la somme des valeurs impaires du tableau T
\item
  \textbf{Réponse 4 :} la somme des valeurs paires du tableau T
\end{enumerate}

    \textbf{Corrigé} : La réponse est la 4.

    \begin{Verbatim}[commandchars=\\\{\}]
{\color{incolor}In [{\color{incolor}21}]:} \PY{k}{def} \PY{n+nf}{mystere}\PY{p}{(}\PY{n}{T}\PY{p}{)}\PY{p}{:}
             \PY{n}{s} \PY{o}{=} \PY{l+m+mi}{0}
             \PY{k}{for} \PY{n}{k} \PY{o+ow}{in} \PY{n}{T}\PY{p}{:}
                 \PY{k}{if} \PY{n}{k} \PY{o}{\PYZpc{}} \PY{l+m+mi}{2} \PY{o}{==} \PY{l+m+mi}{0}\PY{p}{:}
                     \PY{n}{s} \PY{o}{=} \PY{n}{s}\PY{o}{+}\PY{n}{k}
             \PY{k}{return} \PY{n}{s}
         
         \PY{k}{assert} \PY{n}{mystere}\PY{p}{(}\PY{p}{[}\PY{l+m+mi}{1}\PY{p}{,} \PY{l+m+mi}{2}\PY{p}{,} \PY{l+m+mi}{5}\PY{p}{,} \PY{l+m+mi}{4}\PY{p}{]}\PY{p}{)} \PY{o}{==} \PY{l+m+mi}{6}\PY{p}{,} \PY{l+s+s2}{\PYZdq{}}\PY{l+s+s2}{mystere([1, 2,5, 4]) doit être égal à 6}\PY{l+s+s2}{\PYZdq{}}
\end{Verbatim}



    % Add a bibliography block to the postdoc
    
    
    
    \end{document}
